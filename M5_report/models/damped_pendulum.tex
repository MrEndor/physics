\subsection{Свободные колебания с трением}

\textbf{Модель трения}

При наличии трения на маятник действует момент силы трения, который препятствует движению. Рассмотрим модель \textbf{вязкого трения}, в которой момент трения пропорционален угловой скорости:
\[
M_{\text{тр}} = -\gamma I_O \dot{\varphi}
\]

где $\gamma$ -- коэффициент затухания (размерность с$^{-1}$), $I_O$ -- момент инерции.

\textbf{Уравнение движения с трением:}
\[
I_O \ddot{\varphi} = -mgl\sin\varphi - \gamma I_O\dot{\varphi}
\]

или
\[
\boxed{\ddot{\varphi} + 2\beta\dot{\varphi} + \omega_0^2\sin\varphi = 0}
\]

где $\beta = \dfrac{\gamma}{2}$ -- коэффициент затухания, $\omega_0 = \sqrt{\dfrac{mgl}{I_O}}$ -- собственная частота без трения.

\textbf{Линеаризация для малых колебаний}

При малых углах $\sin\varphi \approx \varphi$:
\[
\boxed{\ddot{\varphi} + 2\beta\dot{\varphi} + \omega_0^2\varphi = 0}
\]

Это уравнение затухающего гармонического осциллятора.

\textbf{Общее решение линеаризованного уравнения}

Характеристическое уравнение:
\[
\lambda^2 + 2\beta\lambda + \omega_0^2 = 0
\]

Дискриминант:
\[
D = 4\beta^2 - 4\omega_0^2 = 4(\beta^2 - \omega_0^2)
\]

Рассмотрим случай \textbf{слабого затухания} ($\beta < \omega_0$), наиболее важный для колебаний:

Корни характеристического уравнения:
\[
\lambda_{1,2} = -\beta \pm i\omega_d
\]

где $\omega_d = \sqrt{\omega_0^2 - \beta^2}$ -- частота затухающих колебаний.

Общее решение:
\[
\varphi(t) = Ae^{-\beta t}\cos(\omega_d t + \phi_0)
\]

Для начальных условий $\varphi(0) = \varphi_0$, $\dot{\varphi}(0) = 0$:
\[
A = \varphi_0\sqrt{1 + \dfrac{\beta^2}{\omega_d^2}} \approx \varphi_0, \quad \tg\varphi_0 = \dfrac{\beta}{\omega_d}
\]

При малом затухании ($\beta \ll \omega_0$):
\[
\varphi(t) \approx \varphi_0 e^{-\beta t}\cos(\omega_0 t)
\]

\textbf{Амплитуда затухающих колебаний:}
\[
A(t) = A_0 e^{-\beta t}
\]

где $A_0 = \varphi_0$ -- начальная амплитуда.

\textbf{Период затухающих колебаний:}
\[
T_d = \dfrac{2\pi}{\omega_d} = \dfrac{2\pi}{\sqrt{\omega_0^2 - \beta^2}} = T_0\dfrac{1}{\sqrt{1 - (\beta/\omega_0)^2}}
\]

При малом трении ($\beta \ll \omega_0$):
\[
T_d \approx T_0\left(1 + \dfrac{1}{2}\left(\dfrac{\beta}{\omega_0}\right)^2\right) \approx T_0
\]

Таким образом, \textbf{слабое трение незначительно изменяет период колебаний}, но приводит к экспоненциальному затуханию амплитуды.

\textbf{Зависимость периода от коэффициента трения}

При малых $\beta$:
\[
T_d = T_0\sqrt{\dfrac{1}{1 - \beta^2/\omega_0^2}} \approx T_0\left(1 + \dfrac{\beta^2}{2\omega_0^2}\right)
\]

Период колебаний слабо увеличивается с ростом коэффициента трения, но эффект значительно слабее, чем зависимость от амплитуды в случае без трения.

\textbf{Переходные режимы}

\begin{enumerate}
    \item \textbf{Слабое затухание} ($\beta < \omega_0$): колебательный режим с экспоненциальным затуханием амплитуды
    \item \textbf{Критическое затухание} ($\beta = \omega_0$): апериодическое движение, самое быстрое возвращение к равновесию
    \item \textbf{Сильное затухание} ($\beta > \omega_0$): медленное апериодическое возвращение к равновесию
\end{enumerate}

В нашей работе рассматривается только случай слабого затухания, характерный для реальных маятников.
