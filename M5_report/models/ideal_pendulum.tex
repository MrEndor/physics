\subsection{Свободные колебания без трения}

\textbf{Уравнение движения}

Применим основное уравнение динамики вращательного движения относительно оси вращения $O$:
\[
I_O \ddot{\varphi} = M
\]

где $M$ -- момент всех сил относительно оси $O$.

На маятник действует сила тяжести $\vec{F}_g = m\vec{g}$, приложенная к центру масс. Плечо этой силы относительно оси вращения равно $l \sin\varphi$, где $\varphi$ -- угол отклонения от вертикали (положения равновесия).

Момент силы тяжести:
\[
M = -mgl \sin\varphi
\]

Знак минус появляется, так как сила стремится вернуть маятник в положение равновесия (момент направлен против увеличения угла $\varphi$).

\textbf{Дифференциальное уравнение колебаний:}
\[
\boxed{I_O \ddot{\varphi} + mgl \sin\varphi = 0}
\]

или
\[
\boxed{\ddot{\varphi} + \omega_0^2 \sin\varphi = 0}
\]

где введена \textbf{собственная циклическая частота}:
\[
\omega_0 = \sqrt{\dfrac{mgl}{I_O}}
\]

\textbf{Приближение малых колебаний}

Для малых углов отклонения ($|\varphi| \ll 1$ рад) справедливо разложение:
\[
\sin\varphi \approx \varphi - \dfrac{\varphi^3}{6} + O(\varphi^5) \approx \varphi
\]

Подставляя это в уравнение движения, получаем \textbf{линейное уравнение}:
\[
\boxed{\ddot{\varphi} + \omega_0^2 \varphi = 0}
\]

Это уравнение гармонического осциллятора с общим решением:
\[
\varphi(t) = A\cos(\omega_0 t + \varphi_0)
\]

где $A$ -- амплитуда колебаний, $\varphi_0$ -- начальная фаза.

Для начальных условий $\varphi(0) = \varphi_0$, $\dot{\varphi}(0) = 0$ получаем:
\[
\varphi(t) = \varphi_0 \cos(\omega_0 t)
\]

\textbf{Период малых колебаний:}
\[
T_0 = \dfrac{2\pi}{\omega_0} = 2\pi\sqrt{\dfrac{I_O}{mgl}}
\]

\textbf{Для однородного стержня длины $L$:}
\[
\omega_0 = \sqrt{\dfrac{mg \cdot \frac{L}{2}}{\frac{mL^2}{3}}} = \sqrt{\dfrac{3g}{2L}}, \quad T_0 = 2\pi\sqrt{\dfrac{2L}{3g}}
\]

\textbf{Энергия системы}

Полная механическая энергия физического маятника складывается из кинетической энергии вращения и потенциальной энергии в поле тяжести:
\[
E = \dfrac{1}{2}I_O\dot{\varphi}^2 + mgl(1 - \cos\varphi)
\]

Потенциальная энергия отсчитывается от положения равновесия (вертикаль вниз), где центр масс находится на минимальной высоте. При отклонении на угол $\varphi$ центр масс поднимается на высоту $h = l(1 - \cos\varphi)$.

При малых углах: $1 - \cos\varphi \approx \dfrac{\varphi^2}{2}$, и энергия:
\[
E \approx \dfrac{1}{2}I_O\dot{\varphi}^2 + \dfrac{1}{2}mgl\varphi^2 = \dfrac{1}{2}I_O\omega_0^2\varphi_0^2 = \text{const}
\]

\textbf{Зависимость периода от амплитуды}

Для конечных амплитуд колебаний период зависит от начального угла $\varphi_0$. Точное выражение для периода получается через эллиптический интеграл первого рода:
\[
T(\varphi_0) = 4\sqrt{\dfrac{I_O}{mgl}} \int_0^{\pi/2} \dfrac{d\psi}{\sqrt{1 - k^2\sin^2\psi}}
\]

где $k = \sin\dfrac{\varphi_0}{2}$ -- модуль эллиптического интеграла.

Разложение в ряд по малому параметру $k$:
\[
T(\varphi_0) = T_0\left(1 + \dfrac{1}{4}\sin^2\dfrac{\varphi_0}{2} + \dfrac{9}{64}\sin^4\dfrac{\varphi_0}{2} + O(\varphi_0^6)\right)
\]

Или приближённо:
\[
T(\varphi_0) \approx T_0\left(1 + \dfrac{\varphi_0^2}{16}\right)
\]

Таким образом, с ростом амплитуды период колебаний увеличивается.
