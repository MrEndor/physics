\section{Численные методы решения}

\textbf{Постановка задачи Коши}

Для численного решения преобразуем дифференциальное уравнение второго порядка в систему двух уравнений первого порядка. Введём переменные:
\[
\omega = \dot{\varphi}, \quad \varepsilon = \ddot{\varphi}
\]

\textbf{Система для идеального маятника (без трения):}
\[
\begin{cases}
\dfrac{d\varphi}{dt} = \omega \\[1em]
\dfrac{d\omega}{dt} = -\omega_0^2 \sin\varphi
\end{cases}
\]

где $\omega_0 = \sqrt{\dfrac{mgl}{I_O}}$ -- собственная частота.

\textbf{Система для маятника с трением:}
\[
\begin{cases}
\dfrac{d\varphi}{dt} = \omega \\[1em]
\dfrac{d\omega}{dt} = -\omega_0^2 \sin\varphi - 2\beta\omega
\end{cases}
\]

где $\beta = \dfrac{\gamma}{2}$ -- коэффициент затухания.

\textbf{Начальные условия:}
\[
\varphi(0) = \varphi_0, \quad \omega(0) = \omega_0
\]

Обычно рассматриваем случай $\omega_0 = 0$ (отклонили и отпустили).

\textbf{Метод Рунге-Кутты 4-го порядка}

Для численного интегрирования системы ОДУ используем классический метод Рунге-Кутты 4-го порядка. Это явный одношаговый метод, обеспечивающий точность порядка $O(h^5)$ на одном шаге интегрирования.

\textbf{Общая схема:}

Для системы $\dfrac{d\vec{y}}{dt} = \vec{f}(t, \vec{y})$ с начальными условиями $\vec{y}(t_0) = \vec{y}_0$:

\begin{enumerate}
    \item Вычисляем четыре коэффициента на каждом шаге:
    \begin{align*}
        \vec{k}_1 &= h \cdot \vec{f}(t_n, \vec{y}_n) \\
        \vec{k}_2 &= h \cdot \vec{f}\left(t_n + \dfrac{h}{2}, \vec{y}_n + \dfrac{\vec{k}_1}{2}\right) \\
        \vec{k}_3 &= h \cdot \vec{f}\left(t_n + \dfrac{h}{2}, \vec{y}_n + \dfrac{\vec{k}_2}{2}\right) \\
        \vec{k}_4 &= h \cdot \vec{f}(t_n + h, \vec{y}_n + \vec{k}_3)
    \end{align*}

    \item Обновляем решение по формуле:
    \[
    \vec{y}_{n+1} = \vec{y}_n + \dfrac{1}{6}(\vec{k}_1 + 2\vec{k}_2 + 2\vec{k}_3 + \vec{k}_4)
    \]

    \item Увеличиваем время: $t_{n+1} = t_n + h$
\end{enumerate}

\textbf{Применение к системе маятника:}

Обозначим $\vec{y} = \begin{pmatrix} \varphi \\ \omega \end{pmatrix}$, тогда $\vec{f}(t, \vec{y}) = \begin{pmatrix} \omega \\ -\omega_0^2\sin\varphi - 2\beta\omega \end{pmatrix}$.

На каждом шаге интегрирования:

\begin{align*}
k_{1\varphi} &= h \cdot \omega_n \\
k_{1\omega} &= h \cdot (-\omega_0^2\sin\varphi_n - 2\beta\omega_n) \\[0.5em]
k_{2\varphi} &= h \cdot (\omega_n + k_{1\omega}/2) \\
k_{2\omega} &= h \cdot \left(-\omega_0^2\sin(\varphi_n + k_{1\varphi}/2) - 2\beta(\omega_n + k_{1\omega}/2)\right) \\[0.5em]
k_{3\varphi} &= h \cdot (\omega_n + k_{2\omega}/2) \\
k_{3\omega} &= h \cdot \left(-\omega_0^2\sin(\varphi_n + k_{2\varphi}/2) - 2\beta(\omega_n + k_{2\omega}/2)\right) \\[0.5em]
k_{4\varphi} &= h \cdot (\omega_n + k_{3\omega}) \\
k_{4\omega} &= h \cdot \left(-\omega_0^2\sin(\varphi_n + k_{3\varphi}) - 2\beta(\omega_n + k_{3\omega})\right)
\end{align*}

Обновление:
\[
\varphi_{n+1} = \varphi_n + \dfrac{1}{6}(k_{1\varphi} + 2k_{2\varphi} + 2k_{3\varphi} + k_{4\varphi})
\]
\[
\omega_{n+1} = \omega_n + \dfrac{1}{6}(k_{1\omega} + 2k_{2\omega} + 2k_{3\omega} + k_{4\omega})
\]

\textbf{Выбор шага интегрирования}

Для обеспечения точности и устойчивости численного решения необходимо правильно выбрать шаг интегрирования $h$.

\begin{itemize}
    \item Шаг должен быть значительно меньше периода колебаний: $h \ll T_0$
    \item Чем больше амплитуда, тем меньше должен быть шаг для учёта нелинейности
\end{itemize}

\textbf{Определение периода колебаний}

Период колебаний определяем по времени между последовательными прохождениями маятника через положение равновесия в одном направлении:
\begin{itemize}
    \item Фиксируем моменты времени $t_i$, когда $\varphi(t_i) = 0$ и $\omega(t_i) < 0$ (движение вниз)
    \item Период: $T = t_{i+1} - t_i$
\end{itemize}

Альтернативно, период можно определить как удвоенное время между двумя последовательными точками максимального отклонения ($\omega = 0$).

\newpage
