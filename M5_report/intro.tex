\newpage

\section{Введение}

\Task Моделирование колебаний физического маятника -- твёрдого тела, подвешенного в одной точке и совершающего вращательные движения в вертикальной плоскости под действием силы тяжести. Исследование свободных колебаний без трения и с трением, изучение зависимости периода от амплитуды и коэффициента трения, проверка законов сохранения энергии.

\Goal составить программу для численного решения дифференциального уравнения вращательного движения физического маятника, получить зависимость угла отклонения от времени, исследовать влияние начальной амплитуды и коэффициента трения на период колебаний, сравнить результаты численного моделирования с аналитической теорией малых колебаний.

\section{Физическая постановка задачи}

Рассмотрим твёрдое тело произвольной формы массы $m$, которое может свободно вращаться вокруг неподвижной горизонтальной оси $O$. Под действием силы тяжести тело будет совершать колебания относительно положения равновесия.

\textbf{Основные параметры системы:}

\begin{enumerate}
    \item \textbf{Геометрические параметры:}
    \begin{itemize}
        \item $l$ -- расстояние от оси вращения $O$ до центра масс $C$
        \item $I_O$ -- момент инерции тела относительно оси вращения
        \item $I_C$ -- момент инерции тела относительно оси, проходящей через центр масс
    \end{itemize}

    \item \textbf{Физические параметры:}
    \begin{itemize}
        \item $m$ -- масса тела
        \item $g$ -- ускорение свободного падения ($g \approx 9.81$ м/с$^2$)
        \item $\gamma$ -- коэффициент трения (для модели с трением)
    \end{itemize}

    \item \textbf{Динамические переменные:}
    \begin{itemize}
        \item $\varphi(t)$ -- угол отклонения от положения равновесия
        \item $\omega(t) = \dot{\varphi}(t)$ -- угловая скорость
        \item $\varepsilon(t) = \ddot{\varphi}(t)$ -- угловое ускорение
    \end{itemize}
\end{enumerate}

\textbf{Теорема Штейнера:}

Момент инерции относительно произвольной оси, параллельной оси, проходящей через центр масс, равен:
\[
I_O = I_C + ml^2
\]

\textbf{Для простого физического маятника} (стержень длины $L$ с точечной массой на конце):
\begin{itemize}
    \item Расстояние до центра масс: $l = L$
    \item Момент инерции: $I_O = mL^2$ (пренебрегая массой стержня)
\end{itemize}

\textbf{Для однородного стержня} длины $L$ массы $m$:
\begin{itemize}
    \item Расстояние до центра масс: $l = \dfrac{L}{2}$
    \item Момент инерции относительно центра: $I_C = \dfrac{mL^2}{12}$
    \item Момент инерции относительно конца: $I_O = I_C + ml^2 = \dfrac{mL^2}{12} + m\left(\dfrac{L}{2}\right)^2 = \dfrac{mL^2}{3}$
\end{itemize}

\textbf{Начальные условия} при $t = 0$:
\[
\varphi(0) = \varphi_0, \quad \dot{\varphi}(0) = \omega_0
\]

где $\varphi_0$ -- начальный угол отклонения, $\omega_0$ -- начальная угловая скорость (обычно $\omega_0 = 0$ для свободных колебаний).

\vspace{1cm}

\newpage
