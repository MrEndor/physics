\section{Заключение}

В работе проведено комплексное численное моделирование колебаний физического маятника -- однородного стержня, подвешенного за один конец в поле тяжести. Исследованы две основные модели:

\begin{enumerate}
    \item \textbf{Идеальный маятник без трения} -- консервативная система с сохранением полной механической энергии
    \item \textbf{Маятник с вязким трением} -- диссипативная система с экспоненциальным затуханием амплитуды
\end{enumerate}

\subsection{Физическая интерпретация}

\textbf{Нелинейность и ангармонизм:}

При увеличении амплитуды колебаний проявляется нелинейность возвращающей силы. Маятник дольше находится вблизи крайних положений (где скорость мала) и быстрее проходит через положение равновесия (где скорость максимальна). Это приводит к увеличению периода колебаний.

\textbf{Роль трения:}

Вязкое трение приводит к непрерывной диссипации энергии, пропорциональной квадрату скорости. Это вызывает экспоненциальное затухание амплитуды, но слабо влияет на частоту колебаний при малых коэффициентах трения.

\textbf{Практическое значение:}

Результаты работы применимы к широкому классу колебательных систем: от маятниковых часов до сейсмографов и систем виброзащиты. Понимание зависимости периода от амплитуды важно для точных измерений времени, а учёт затухания необходим для расчёта демпфирующих систем.

\subsection{Итог}

Разработанная программа численного моделирования позволяет:
\begin{itemize}
    \item Точно решать нелинейное уравнение колебаний физического маятника
    \item Исследовать влияние начальных условий и параметров системы на характер движения
    \item Проверять выполнение законов сохранения и сравнивать с аналитической теорией
    \item Визуализировать траектории, фазовые портреты и временны́е зависимости
\end{itemize}

