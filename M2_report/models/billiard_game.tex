\begin{center}
    \subsection{Дополнительное задание: Упрощённая игра в бильярд}
\end{center}

\begin{center}
    \textbf{Постановка задачи}
\end{center}

Реализация алгоритма выбора направления удара по битку для попадания прицельного шара в заданную лузу.

\begin{center}
    \textbf{Геометрия задачи}
\end{center}

Пусть:
\begin{itemize}
    \item $\vec{r}_b$ -- позиция битка (белый шар)
    \item $\vec{r}_t$ -- позиция прицельного шара (цветной шар)
    \item $\vec{r}_p$ -- позиция лузы (цель)
    \item $R$ -- радиус шаров
\end{itemize}

\begin{center}
    \textbf{Алгоритм расчёта направления удара}
\end{center}

\textit{Шаг 1: Определение точки контакта}

Для попадания прицельного шара в лузу, биток должен ударить его таким образом, чтобы линия центров в момент контакта была направлена к лузе.

Точка контакта $\vec{r}_c$ (позиция центра битка в момент столкновения) находится на линии, соединяющей центр прицельного шара с лузой, на расстоянии $2R$ от центра прицельного шара:
\[
\vec{r}_c = \vec{r}_t - 2R \frac{\vec{r}_p - \vec{r}_t}{|\vec{r}_p - \vec{r}_t|}
\]

\textit{Шаг 2: Направление удара}

Направление удара битка определяется вектором от центра битка к точке контакта:
\[
\vec{d} = \frac{\vec{r}_c - \vec{r}_b}{|\vec{r}_c - \vec{r}_b|}
\]

\textit{Шаг 3: Проверка возможности удара}

Удар возможен, если:
\begin{enumerate}
    \item Расстояние от битка до точки контакта больше $2R$: $|\vec{r}_c - \vec{r}_b| > 2R$
    \item На траектории битка нет препятствий (других шаров)
    \item Угол удара не слишком острый (практическое ограничение)
\end{enumerate}

\begin{center}
    \textbf{Расчёт необходимой скорости}
\end{center}

Для центрального удара (биток попадает точно в центр прицельного шара) при равных массах:
\[
v_{\text{биток}}^{\text{после}} = 0, \quad v_{\text{прицел}}^{\text{после}} = v_{\text{биток}}^{\text{до}}
\]

Для нецентрального удара скорость прицельного шара после столкновения:
\[
v_{\text{прицел}} = v_{\text{биток}} \cos\theta
\]
где $\theta$ -- угол между линией центров и направлением движения битка.

\begin{center}
    \textbf{Алгоритм поиска оптимального угла}
\end{center}

При наличии нескольких возможных траекторий:
\begin{enumerate}
    \item Рассчитать все возможные точки контакта для попадания в лузу
    \item Для каждой точки проверить достижимость и отсутствие препятствий
    \item Выбрать траекторию с минимальным углом отклонения или минимальной силой удара
    \item Рассчитать необходимые параметры удара (направление и сила)
\end{enumerate}

\begin{center}
    \textbf{Учёт отражений от бортов}
\end{center}    

Для более сложных траекторий с отражениями:
\begin{itemize}
    \item Использовать принцип зеркального отражения
    \item Построить виртуальные изображения лузы относительно бортов
    \item Рассчитать прямую траекторию к виртуальной лузе
    \item Проверить, что траектория проходит через допустимые области стола
\end{itemize}