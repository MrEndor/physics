\begin{center}
    \subsection{Метод 1: Применение законов сохранения}
\end{center}

\begin{center}
    \textbf{Столкновение со стенкой}
\end{center}

При столкновении с вертикальной или горизонтальной стенкой компонента скорости, перпендикулярная стенке, меняет знак, а параллельная остаётся неизменной.

Для вертикальной стенки (в точке $x = x_{\text{wall}}$):
\[
\begin{cases}
v_x' = -v_x \\
v_y' = v_y
\end{cases}
\]

Для горизонтальной стенки (в точке $y = y_{\text{wall}}$):
\[
\begin{cases}
v_x' = v_x \\
v_y' = -v_y
\end{cases}
\]

\begin{center}
    \textbf{Столкновение двух шаров}
\end{center}

Рассмотрим столкновение двух шаров с массами $m_1, m_2$ и скоростями $\vec{v}_1, \vec{v}_2$ до столкновения.

\textit{Система уравнений:}
\begin{itemize}
    \item \textbf{Закон сохранения импульса:}
    \[
    m_1\vec{v}_1 + m_2\vec{v}_2 = m_1\vec{v}_1' + m_2\vec{v}_2'
    \]

    \item \textbf{Закон сохранения энергии:}
    \[
    \frac{1}{2}m_1v_1^2 + \frac{1}{2}m_2v_2^2 = \frac{1}{2}m_1v_1'^2 + \frac{1}{2}m_2v_2'^2
    \]
\end{itemize}

\Solution Введём систему координат, связанную с линией центров в момент столкновения.

Пусть $\vec{n}$ -- единичный вектор, направленный от центра первого шара к центру второго:
\[
\vec{n} = \frac{\vec{r}_2 - \vec{r}_1}{|\vec{r}_2 - \vec{r}_1|}
\]

Разложим скорости на нормальные и тангенциальные компоненты:
\[
\begin{aligned}
v_{1n} &= \vec{v}_1 \cdot \vec{n}, & \quad v_{1\tau} &= \vec{v}_1 - v_{1n}\vec{n} \\
v_{2n} &= \vec{v}_2 \cdot \vec{n}, & \quad v_{2\tau} &= \vec{v}_2 - v_{2n}\vec{n}
\end{aligned}
\]

\newpage
При упругом столкновении:
\begin{itemize}
    \item Тангенциальные компоненты не изменяются: $v_{1\tau}' = v_{1\tau}$, $v_{2\tau}' = v_{2\tau}$
    \item Нормальные компоненты определяются из законов сохранения
\end{itemize}

\begin{center}
    \textbf{Вывод формул для нормальных компонент:}
\end{center}

Применяем законы сохранения только к нормальным компонентам:

\textit{Закон сохранения импульса (нормальная компонента):}
\[
m_1 v_{1n} + m_2 v_{2n} = m_1 v_{1n}' + m_2 v_{2n}' \quad (1)
\]

\textit{Закон сохранения энергии (нормальная компонента):}
\[
\frac{1}{2}m_1 v_{1n}^2 + \frac{1}{2}m_2 v_{2n}^2 = \frac{1}{2}m_1 v_{1n}'^2 + \frac{1}{2}m_2 v_{2n}'^2 \quad (2)
\]

Из уравнения (2) после алгебраических преобразований получаем условие:
\[
v_{1n} - v_{1n}' = -(v_{2n} - v_{2n}') \quad (3)
\]

Решая систему уравнений (1) и (3), получаем:
\[
v_{1n}' = \frac{(m_1 - m_2)v_{1n} + 2m_2 v_{2n}}{m_1 + m_2}
\]
\[
v_{2n}' = \frac{(m_2 - m_1)v_{2n} + 2m_1 v_{1n}}{m_1 + m_2}
\]

Окончательные скорости:
\[
\vec{v}_1' = v_{1n}'\vec{n} + \vec{v}_{1\tau}, \quad \vec{v}_2' = v_{2n}'\vec{n} + \vec{v}_{2\tau}
\]

\textbf{Частный случай: равные массы} ($m_1 = m_2$)
\[
v_{1n}' = v_{2n}, \quad v_{2n}' = v_{1n}
\]
Нормальные компоненты скоростей просто обмениваются.

\textbf{Проверка других частных случаев:}

\textit{Один шар неподвижен} ($v_{2n} = 0$):
\[
v_{1n}' = \frac{m_1 - m_2}{m_1 + m_2}v_{1n}, \quad v_{2n}' = \frac{2m_1}{m_1 + m_2}v_{1n}
\]

При $m_1 \gg m_2$: $v_{1n}' \approx v_{1n}$, $v_{2n}' \approx 2v_{1n}$ (тяжёлый шар почти не изменяет скорость)

При $m_1 \ll m_2$: $v_{1n}' \approx -v_{1n}$, $v_{2n}' \approx 0$ (лёгкий шар отскакивает)