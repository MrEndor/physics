\begin{center}
    \section{Заключение}
\end{center}

В работе проведено комплексное исследование динамики упругих столкновений в системе бильярдных шаров с использованием различных физических моделей и численных методов.

\textbf{Основные достижения работы:}

\begin{enumerate}
    \item \textbf{Реализованы два подхода к моделированию столкновений:}
    \begin{itemize}
        \item Аналитический метод на основе законов сохранения импульса и энергии
        \item Численный метод с детальным моделированием силового взаимодействия
    \end{itemize}

    \item \textbf{Исследована модель контактного взаимодействия:}
    \begin{itemize}
        \item Модель с силой Гука ($F \sim -\Delta x$) -- простая и эффективная линейная модель
    \end{itemize}

    \item \textbf{Разработан эффективный численный алгоритм:}
    \begin{itemize}
        \item Использование метода Рунге-Кутты 4-го порядка для интегрирования ОДУ
        \item Адаптивная детекция столкновений с коррекцией временного шага
    \end{itemize}

    \item \textbf{Создана упрощённая версия игры в бильярд:}
    \begin{itemize}
        \item Алгоритм расчёта оптимального направления удара
        \item Геометрический анализ траекторий с учётом препятствий
        \item Обработка отражений от бортов стола
    \end{itemize}
\end{enumerate}