\begin{center}
    \section{Численные методы решения}
\end{center}

\begin{center}
    \textbf{Численное интегрирование системы ОДУ}
\end{center}

Для решения систем дифференциальных уравнений, возникающих в модели с силой Гука, используем метод Рунге-Кутты 4-го порядка.

\begin{center}
    \textbf{Преобразование к системе первого порядка}
\end{center}

Исходная система уравнений второго порядка:
\[
\begin{cases}
m_1 \ddfrac{d^2\vec{r}_1}{dt^2} = \vec{F}_{12} \\[1em]
m_2 \ddfrac{d^2\vec{r}_2}{dt^2} = -\vec{F}_{12}
\end{cases}
\]

В терминах скоростей:
\[
\vec{v}_1 = \dfrac{d\vec{r}_1}{dt}, \quad \vec{v}_2 = \dfrac{d\vec{r}_2}{dt}
\]

Получаем систему первого порядка:
\[
\begin{cases}
\ddfrac{d\vec{r}_1}{dt} = \vec{v}_1 \\[1em]
\ddfrac{d\vec{r}_2}{dt} = \vec{v}_2 \\[1em]
\ddfrac{d\vec{v}_1}{dt} = \dfrac{\vec{F}_{12}}{m_1} \\[1em]
\ddfrac{d\vec{v}_2}{dt} = -\dfrac{\vec{F}_{12}}{m_2}
\end{cases}
\]

\begin{center}
    \textbf{Алгоритм Рунге-Кутты 4-го порядка}
\end{center}

Для системы $\dfrac{d\vec{y}}{dt} = \vec{f}(t, \vec{y})$ с начальным условием $\vec{y}(t_0) = \vec{y}_0$:

\begin{enumerate}
    \item Выбираем шаг интегрирования $\Delta h$
    \item На каждом шаге вычисляем коэффициенты:
    \begin{align}
    \vec{k}_1 &= \Delta h \vec{f}(t_n, \vec{y}_n) \\
    \vec{k}_2 &= \Delta h \vec{f}(t_n + \Delta h/2, \vec{y}_n + \vec{k}_1/2) \\
    \vec{k}_3 &= \Delta h \vec{f}(t_n + \Delta h/2, \vec{y}_n + \vec{k}_2/2) \\
    \vec{k}_4 &= \Delta h \vec{f}(t_n + \Delta h, \vec{y}_n + \vec{k}_3)
    \end{align}
    \item Обновляем решение:
    \[
    \vec{y}_{n+1} = \vec{y}_n + \dfrac{1}{6}(\vec{k}_1 + 2\vec{k}_2 + 2\vec{k}_3 + \vec{k}_4)
    \]
\end{enumerate}

\begin{center}
    \textbf{Детекция столкновений}
\end{center}

\textit{Проблема:} При использовании фиксированного шага интегрирования может произойти "проскакивание"  момента столкновения.

\textit{Решение:} Адаптивный алгоритм детекции:
\begin{enumerate}
    \item На каждом шаге проверяем условие $|\vec{r}_2 - \vec{r}_1| < 2R$
    \item При пересечении используем бинпоиск для точного определения момента контакта
    \item Корректируем временной шаг для точного попадания в момент столкновения
\end{enumerate}

\begin{center}
    \textbf{Выбор шага интегрирования}
\end{center}

Критерии выбора временного шага $\Delta h$:
\begin{itemize}
    \item \textbf{Точность:} $\Delta h$ должно быть достаточно малым для обеспечения заданной точности
\end{itemize}

\newpage