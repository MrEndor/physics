\newpage

\begin{center}
    \section{Введение}
\end{center}

\Task Моделирование столкновений идеально гладкого бильярдного шара с бесконечно тяжёлой стенкой и с другим шаром двумя различными подходами: аналитическим (с применением законов сохранения) и численным (на основе дифференциальных уравнений).

\Goal составить программу для моделирования упругих столкновений в бильярде, реализовать два метода расчёта, провести их сравнение и создать простейшую версию игры в бильярд с алгоритмом выбора направления удара.

\begin{center}
    \section{Физическая постановка задачи}
\end{center}
\vspace{-2em}

Рассмотрим систему бильярдных шаров радиуса $R$ и массы $m$ на гладкой поверхности стола. В данной работе исследуем следующие виды столкновений:

\begin{enumerate}
    \item \textbf{Столкновение с бесконечно тяжёлой стенкой} - шар отражается от неподвижной стенки
    \item \textbf{Столкновение двух шаров} - взаимодействие между подвижными объектами
\end{enumerate}

\textbf{Основные предположения:}
\begin{itemize}
    \item Шары считаются абсолютно твёрдыми и идеально гладкими
    \item Столкновения абсолютно упругие (сохраняются кинетическая энергия и импульс)
    \item Трение о поверхность стола отсутствует
    \item Вращение шаров не учитывается (поступательное движение)
\end{itemize}

\textbf{Координатная система:}
Используем декартову систему координат с началом в углу стола. Положение $i$-го шара задаётся вектором $\vec{r}_i = (x_i, y_i)$, скорость -- вектором $\vec{v}_i = (v_{ix}, v_{iy})$.

\textbf{Условия столкновения:}
\begin{itemize}
    \item \textit{Со стенкой:} шар касается границы стола
    \item \textit{Между шарами:} расстояние между центрами равно $2R$: $|\vec{r}_1 - \vec{r}_2| = 2R$
\end{itemize}

\newpage