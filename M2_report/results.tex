\begin{center}
    \section{Результаты и анализ}
    \subsection{Сравнение методов расчёта}
\end{center}

\textbf{Валидация численных методов}

Для проверки корректности реализации сравниваем результаты двух подходов:
\begin{enumerate}
    \item Аналитический расчёт по законам сохранения
    \item Численное моделирование с силой Гука
\end{enumerate}

\textbf{Тестовые случаи:}

\begin{enumerate}
    \item \textbf{Центральное столкновение двух одинаковых шаров}
    \begin{itemize}
        \item Начальные условия: $m_1 = m_2 = m$, $\vec{v}_1 = (v_0, 0)$, $\vec{v}_2 = (0, 0)$
        \item Ожидаемый результат: $\vec{v}_1' = (0, 0)$, $\vec{v}_2' = (v_0, 0)$ (полная передача импульса)
    \end{itemize}

    \item \textbf{Нецентральное столкновение под углом $45\degree$}}
    \begin{itemize}
        \item Проверка сохранения компонент импульса и кинетической энергии
    \end{itemize}

    \item \textbf{Столкновение со стенкой}
    \begin{itemize}
        \item Простейший случай для валидации отражений
        \item Точное совпадение с аналитическим решением
    \end{itemize}
\end{enumerate}

\begin{center}
    \subsection{Влияние параметров на динамику столкновений}
\end{center}

\begin{enumerate}
    \item \textbf{Начальная скорость}:
    \begin{itemize}
        \item При увеличении скорости время контакта уменьшается
        \item Максимальная сила растёт пропорционально скорости в модели Гука
    \end{itemize}

    \item \textbf{Угол столкновения}:
    \begin{itemize}
        \item Скользящие удары (большой угол) более чувствительны к численным ошибкам
        \item При углах близких к $90°$ требуется повышенная точность расчётов
    \end{itemize}

    \item \textbf{Соотношение масс}:
    \begin{itemize}
        \item При $m_1 \gg m_2$: лёгкий шар приобретает скорость $\approx 2v_1$
        \item При $m_1 \ll m_2$: тяжёлый шар практически не движется
    \end{itemize}
\end{enumerate}

\newpage