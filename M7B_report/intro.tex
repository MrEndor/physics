\newpage

\section{Введение}

\Task Исследование процесса установления термодинамического равновесия в системе идеального газа, находящегося в вертикальном сосуде в поле тяжести. Рассмотрение неравновесного начального состояния, когда все молекулы сосредоточены вблизи дна сосуда и имеют одинаковую кинетическую энергию.

\Goal
\begin{enumerate}
    \item Провести численное моделирование движения молекул идеального газа в замкнутом вертикальном сосуде с учётом упругих столкновений со стенками и между собой
    \item Получить распределения молекул по скоростям и по высоте в равновесном состоянии
    \item Сравнить полученные распределения с теоретическими: распределением Максвелла по скоростям и барометрической формулой для распределения по высоте
\end{enumerate}

\section{Физическая постановка задачи}
\vspace{-1em}

Рассматривается система из $N$ одинаковых молекул идеального газа массой $m$ каждая, находящихся в вертикальном цилиндрическом сосуде высотой $H$ и площадью основания $S$ в гравитационном поле с ускорением $g$.

\textbf{Модель идеального газа предполагает:}
\begin{enumerate}
    \item Молекулы -- материальные точки (размеры много меньше расстояний между ними)
    \item Столкновения молекул между собой и со стенками абсолютно упругие
    \item Взаимодействие молекул происходит только при столкновениях (потенциальная энергия взаимодействия пренебрежимо мала)
    \item Стенки сосуда непроницаемы и гладкие
\end{enumerate}

\textbf{Начальные условия:}
\begin{itemize}
    \item Все молекулы находятся в тонком слое вблизи дна: $0 < z < h_0 \ll H$
    \item Все молекулы имеют одинаковую по модулю скорость $v_0$, направленную случайным образом
    \item Начальная кинетическая энергия каждой молекулы: $E_0 = \dfrac{mv_0^2}{2}$
\end{itemize}

\textbf{Рассматриваются два режима:}
\begin{enumerate}
    \item \textbf{Изолированная система}: полная энергия $E = const$, стенки теплоизолированы
    \item \textbf{Термостат}: температура стенок $T = const$, энергия системы флуктуирует
\end{enumerate}

\newpage
