\subsection{Распределение Максвелла-Больцмана}

Полное распределение молекул в фазовом пространстве координат и импульсов описывается \textbf{распределением Максвелла-Больцмана}.

\textbf{Каноническое распределение Гиббса}

Вероятность найти систему в состоянии с энергией $E$ при температуре $T$:
\[
P(E) \propto \exp\left(-\dfrac{E}{k_B T}\right)
\]

Это называется \textbf{фактором Больцмана}.

\textbf{Распределение одной молекулы}

Для одной молекулы в гравитационном поле полная энергия:
\[
E(\vec{r}, \vec{v}) = \dfrac{mv^2}{2} + mgz
\]

Плотность вероятности в фазовом пространстве:
\[
f(\vec{r}, \vec{v}) = C \exp\left(-\dfrac{E(\vec{r}, \vec{v})}{k_B T}\right) = C \exp\left(-\dfrac{mv^2/2 + mgz}{k_B T}\right)
\]

где $C$ -- константа нормировки.

\Solution Найдём константу нормировки $C$ из условия:
\[
\int_{V} d^3r \int_{-\infty}^{\infty} d^3v \, f(\vec{r}, \vec{v}) = 1
\]

Разделяя экспоненты:
\[
f(\vec{r}, \vec{v}) = C \exp\left(-\dfrac{mv^2}{2k_B T}\right) \exp\left(-\dfrac{mgz}{k_B T}\right)
\]

Интегрируем по скоростям и координатам отдельно, получая:
\[
C = \left(\dfrac{m}{2\pi k_B T}\right)^{3/2} \cdot \dfrac{mg}{Sk_B T}
\]

где $S$ -- площадь основания сосуда.

Это распределение факторизуется:
\[
f(\vec{r}, \vec{v}) = f(\vec{v}) \cdot f(z)
\]

где:
\begin{itemize}
    \item $f(\vec{v})$ -- распределение Максвелла по скоростям
    \item $f(z)$ -- барометрическое распределение по высоте
\end{itemize}

\textbf{Распределение по энергии}

Полная энергия молекулы $E = \dfrac{mv^2}{2} + mgz$ может принимать значения от 0 до $\infty$.

Распределение по полной энергии в термодинамическом равновесии:
\[
f(E) \propto \sqrt{E} \exp\left(-\dfrac{E}{k_B T}\right)
\]

Множитель $\sqrt{E}$ появляется из-за статистического веса -- числа способов реализовать данную энергию.

\textbf{Средняя энергия молекулы}

В трёхмерном пространстве для молекулы в гравитационном поле:
\[
\langle E \rangle = \left\langle \dfrac{mv^2}{2} \right\rangle + \langle mgz \rangle
\]

Из распределения Максвелла: $\left\langle \dfrac{mv^2}{2} \right\rangle = \dfrac{3}{2}k_B T$

Из барометрической формулы: $\langle mgz \rangle = k_B T$

Итого:
\[
\boxed{\langle E \rangle = \dfrac{3}{2}k_B T + k_B T = \dfrac{5}{2}k_B T}
\]

\textbf{Связь с термодинамикой}

Для системы из $N$ молекул идеального газа:
\begin{itemize}
    \item Внутренняя энергия (без учёта гравитации): $U = N \cdot \dfrac{3}{2}k_B T = \dfrac{3}{2}NkT = \dfrac{3}{2}\nu RT$
    \item Теплоёмкость при постоянном объёме: $C_V = \dfrac{\partial U}{\partial T} = \dfrac{3}{2}\nu R$
    \item Теплоёмкость при постоянном давлении: $C_P = C_V + \nu R = \dfrac{5}{2}\nu R$
    \item Показатель адиабаты: $\gamma = \dfrac{C_P}{C_V} = \dfrac{5/2}{3/2} = \dfrac{5}{3} \approx 1.67$
\end{itemize}

где $\nu$ -- число молей, $R = N_A k_B$ -- универсальная газовая постоянная.
