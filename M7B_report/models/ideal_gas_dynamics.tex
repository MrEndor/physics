\subsection{Динамика молекул идеального газа}

\textbf{Уравнения движения}

Каждая молекула движется под действием силы тяжести согласно \textbf{второму закону Ньютона}:
\[
m\dfrac{d\vec{v}}{dt} = m\vec{g}
\]

В декартовых координатах ($x$, $y$ -- горизонтальные, $z$ -- вертикальная ось):
\[
\begin{cases}
\dfrac{dv_x}{dt} = 0 \\[0.5em]
\dfrac{dv_y}{dt} = 0 \\[0.5em]
\dfrac{dv_z}{dt} = -g
\end{cases}
\quad \text{и} \quad
\begin{cases}
\dfrac{dx}{dt} = v_x \\[0.5em]
\dfrac{dy}{dt} = v_y \\[0.5em]
\dfrac{dz}{dt} = v_z
\end{cases}
\]

Горизонтальные компоненты скорости сохраняются между столкновениями, вертикальная изменяется под действием гравитации.

\textbf{Граничные условия}

Упругое отражение от стенок сосуда:
\begin{itemize}
    \item \textbf{Дно} ($z = 0$): при $z \leq 0$ компонента $v_z \to -v_z$
    \item \textbf{Крышка} ($z = H$): при $z \geq H$ компонента $v_z \to -v_z$
    \item \textbf{Боковые стенки} (цилиндр радиуса $R$): при $\sqrt{x^2 + y^2} \geq R$ нормальная компонента скорости меняет знак
\end{itemize}

Для моделирования термостата при столкновении со стенкой скорость молекулы пересчитывается по распределению Максвелла при заданной температуре стенки $T$.

\textbf{Столкновения между молекулами}

При столкновении двух молекул выполняются законы сохранения:
\begin{enumerate}
    \item \textbf{Сохранение импульса:}
    \[
    m\vec{v}_1 + m\vec{v}_2 = m\vec{v}_1' + m\vec{v}_2'
    \]

    \item \textbf{Сохранение кинетической энергии:}
    \[
    \dfrac{m v_1^2}{2} + \dfrac{m v_2^2}{2} = \dfrac{m v_1'^2}{2} + \dfrac{m v_2'^2}{2}
    \]
\end{enumerate}

где $\vec{v}_1, \vec{v}_2$ -- скорости до столкновения, $\vec{v}_1', \vec{v}_2'$ -- после столкновения.

Столкновение происходит при сближении молекул на расстояние меньше эффективного диаметра $d$.

\textbf{Алгоритм расчёта скоростей после столкновения:}

Переход в систему центра масс:
\[
\vec{V}_{cm} = \dfrac{\vec{v}_1 + \vec{v}_2}{2}
\]

Относительная скорость:
\[
\vec{v}_{rel} = \vec{v}_1 - \vec{v}_2
\]

Единичный вектор вдоль линии центров:
\[
\vec{n} = \dfrac{\vec{r}_1 - \vec{r}_2}{|\vec{r}_1 - \vec{r}_2|}
\]

Новые скорости после упругого столкновения:
\[
\begin{cases}
\vec{v}_1' = \vec{V}_{cm} + \dfrac{1}{2}\left[\vec{v}_{rel} - 2(\vec{v}_{rel} \cdot \vec{n})\vec{n}\right] \\[1em]
\vec{v}_2' = \vec{V}_{cm} - \dfrac{1}{2}\left[\vec{v}_{rel} - 2(\vec{v}_{rel} \cdot \vec{n})\vec{n}\right]
\end{cases}
\]

\textbf{Полная энергия системы}

В изолированной системе полная энергия сохраняется:
\[
E_{total} = \sum_{i=1}^{N} \left(\dfrac{mv_i^2}{2} + mgz_i\right) = const
\]

где первое слагаемое -- кинетическая энергия, второе -- потенциальная энергия в поле тяжести.
