\subsection{Барометрическая формула}

В состоянии термодинамического равновесия газ в гравитационном поле имеет неоднородное распределение плотности по высоте.

\textbf{Вывод барометрической формулы}

Рассмотрим элемент газа между высотами $z$ и $z + dz$. Условие механического равновесия:
\[
p(z + dz) \cdot S = p(z) \cdot S + \rho(z) g S dz
\]

где $p(z)$ -- давление на высоте $z$, $\rho(z)$ -- плотность массы.

Дифференцируя:
\[
\dfrac{dp}{dz} = -\rho(z) g
\]

Используем уравнение состояния идеального газа:
\[
p = \dfrac{\rho}{m} k_B T = n(z) k_B T
\]

где $n(z) = \dfrac{\rho(z)}{m}$ -- концентрация молекул (число молекул на единицу объёма).

Подставляем в условие равновесия:
\[
\dfrac{d(n k_B T)}{dz} = -n(z) m g
\]

Для изотермической атмосферы ($T = const$):
\[
\dfrac{dn}{dz} = -\dfrac{mg}{k_B T} n(z)
\]

\Solution Разделяем переменные и интегрируем:
\[
\dfrac{dn}{n} = -\dfrac{mg}{k_B T} dz
\]
\[
\ln n(z) - \ln n(0) = -\dfrac{mgz}{k_B T}
\]

Получаем \textbf{барометрическую формулу}:
\[
\boxed{n(z) = n(0) \exp\left(-\dfrac{mgz}{k_B T}\right)}
\]

или в терминах давления:
\[
p(z) = p(0) \exp\left(-\dfrac{mgz}{k_B T}\right)
\]

\textbf{Характерная высота атмосферы}

Введём масштаб высоты:
\[
H_0 = \dfrac{k_B T}{mg}
\]

Тогда барометрическая формула принимает вид:
\[
n(z) = n(0) e^{-z/H_0}
\]

\textbf{Распределение молекул по высоте}

Вероятность найти молекулу на высоте между $z$ и $z + dz$ пропорциональна концентрации:
\[
\boxed{f(z) = \dfrac{1}{H_0} \exp\left(-\dfrac{z}{H_0}\right)}
\]

где нормировка:
\[
\int_0^\infty f(z) dz = \int_0^\infty \dfrac{1}{H_0} e^{-z/H_0} dz = 1
\]

\Solution Проверим нормировку:
\[
\int_0^\infty \dfrac{1}{H_0} e^{-z/H_0} dz = \dfrac{1}{H_0} \int_0^\infty e^{-z/H_0} dz
\]

Замена переменной: $u = \dfrac{z}{H_0}$, $dz = H_0 du$:
\[
\dfrac{1}{H_0} \int_0^\infty e^{-z/H_0} dz = \dfrac{1}{H_0} \cdot H_0 \int_0^\infty e^{-u} du = \left[-e^{-u}\right]_0^\infty = 0 - (-1) = 1
\]

\textbf{Средняя высота молекулы}

\Solution Вычислим среднюю высоту:
\[
\langle z \rangle = \int_0^\infty z f(z) dz = \int_0^\infty \dfrac{z}{H_0} e^{-z/H_0} dz
\]

Замена переменной: $u = \dfrac{z}{H_0}$, $z = H_0 u$, $dz = H_0 du$:
\[
\langle z \rangle = \int_0^\infty \dfrac{H_0 u}{H_0} e^{-u} H_0 du = H_0 \int_0^\infty u e^{-u} du
\]

Интегрирование по частям или используя табличный интеграл:
\[
\int_0^\infty u e^{-u} du = \Gamma(2) = 1! = 1
\]

Таким образом:
\[
\langle z \rangle = H_0 \cdot 1 = H_0 = \dfrac{k_B T}{mg}
\]

\textbf{Потенциальная энергия в среднем}

\[
\langle U \rangle = \langle mgz \rangle = mg \langle z \rangle = mg \cdot \dfrac{k_B T}{mg} = k_B T
\]

Это согласуется с теоремой о равнораспределении: на каждую "квадратичную"  координату приходится энергия $\dfrac{1}{2}k_B T$, но в гравитационном поле потенциальная энергия линейна по высоте, и среднее значение равно $k_B T$.

\textbf{График барометрического распределения}

\begin{figure}[h]
\centering
\begin{tikzpicture}
\begin{axis}[
    width=0.8\textwidth,
    height=7cm,
    xlabel={Высота $z/H_0$},
    ylabel={Относительная концентрация $n(z)/n(0)$},
    grid=major,
    domain=0:5,
    ymin=0,
    ymax=1,
    samples=200,
]
% Барометрическая формула: n(z)/n(0) = exp(-z/H_0)
\addplot[blue, thick] {exp(-x)};
\addlegendentry{$n(z)/n(0) = e^{-z/H_0}$}

% Отмечаем характерные высоты
\draw[dashed, gray] (axis cs:0,0.368) -- (axis cs:5,0.368);
\draw[dashed, gray] (axis cs:1,0) -- (axis cs:1,1);
\node[anchor=west, gray] at (axis cs:1.1,0.368) {$e^{-1} \approx 0.37$};
\node[anchor=north, gray] at (axis cs:1,0.05) {$H_0$};
\end{axis}
\end{tikzpicture}
\caption{Барометрическое распределение в безразмерных единицах. На высоте $z = H_0$ концентрация уменьшается в $e$ раз. Для воздуха при комнатной температуре $H_0 \approx 8.5$ км}
\end{figure}

Физический смысл:
\begin{itemize}
    \item Экспоненциальное убывание концентрации с высотой
    \item Характерная высота атмосферы $H_0 = k_BT/(mg)$ зависит от температуры и массы молекул
    \item Для воздуха при комнатной температуре $H_0 \approx 8.5$ км
    \item Более лёгкие газы (водород, гелий) имеют большую высоту шкалы
\end{itemize}
