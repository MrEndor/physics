\subsection{Распределение Максвелла по скоростям}

В состоянии термодинамического равновесия распределение молекул по скоростям описывается \textbf{распределением Максвелла}.

\textbf{Распределение по вектору скорости}

Плотность вероятности обнаружить молекулу со скоростью в интервале $(\vec{v}, \vec{v} + d\vec{v})$:
\[
f(\vec{v}) = \left(\dfrac{m}{2\pi k_B T}\right)^{3/2} \exp\left(-\dfrac{m(\vec{v})^2}{2k_B T}\right)
\]

где:
\begin{itemize}
    \item $m$ -- масса молекулы
    \item $k_B = 1.38 \times 10^{-23}$ Дж/К -- постоянная Больцмана
    \item $T$ -- абсолютная температура
\end{itemize}

Распределение факторизуется по компонентам:
\[
f(\vec{v}) = f(v_x)f(v_y)f(v_z)
\]

где каждая компонента имеет нормальное распределение:
\[
f(v_i) = \sqrt{\dfrac{m}{2\pi k_B T}} \exp\left(-\dfrac{mv_i^2}{2k_B T}\right), \quad i = x, y, z
\]

\textbf{Распределение по модулю скорости}

Интегрируя по всем направлениям, получаем распределение по модулю скорости $v = |\vec{v}|$:
\[
\boxed{f(v) = 4\pi v^2 \left(\dfrac{m}{2\pi k_B T}\right)^{3/2} \exp\left(-\dfrac{mv^2}{2k_B T}\right)}
\]

Это распределение называется \textbf{распределением Максвелла}.

\textbf{Характерные скорости}

\begin{enumerate}
    \item \textbf{Наиболее вероятная скорость} (максимум распределения):

    Найдём максимум $f(v)$, приравняв производную к нулю:
    \[
    \dfrac{df}{dv} = 4\pi \left(\dfrac{m}{2\pi k_B T}\right)^{3/2} \dfrac{d}{dv}\left[v^2 e^{-mv^2/(2k_B T)}\right] = 0
    \]

    Применяя правило произведения:
    \[
    2v e^{-mv^2/(2k_B T)} + v^2 e^{-mv^2/(2k_B T)} \cdot \left(-\dfrac{mv}{k_B T}\right) = 0
    \]

    Вынося общий множитель $v e^{-mv^2/(2k_B T)}$:
    \[
    v e^{-mv^2/(2k_B T)} \left(2 - \dfrac{mv^2}{k_B T}\right) = 0
    \]

    Нетривиальное решение ($v \neq 0$):
    \[
    2 - \dfrac{mv^2}{k_B T} = 0 \quad \Rightarrow \quad v_{prob} = \sqrt{\dfrac{2k_B T}{m}}
    \]

    \item \textbf{Средняя скорость:}
    \[
    \langle v \rangle = \int_0^\infty v f(v) dv = 4\pi \left(\dfrac{m}{2\pi k_B T}\right)^{3/2} \int_0^\infty v^3 e^{-mv^2/(2k_B T)} dv
    \]

    Замена переменной: $u = \dfrac{mv^2}{2k_B T}$, откуда $v = \sqrt{\dfrac{2k_B T u}{m}}$, $dv = \sqrt{\dfrac{k_B T}{2mu}} du$
    \[
    \int_0^\infty v^3 e^{-mv^2/(2k_B T)} dv = \left(\dfrac{2k_B T}{m}\right)^2 \int_0^\infty u e^{-u} du = \left(\dfrac{2k_B T}{m}\right)^2 \cdot 1! = \left(\dfrac{2k_B T}{m}\right)^2
    \]

    Подставляя обратно:
    \[
    \langle v \rangle = 4\pi \left(\dfrac{m}{2\pi k_B T}\right)^{3/2} \left(\dfrac{2k_B T}{m}\right)^2 = \sqrt{\dfrac{8k_B T}{\pi m}}
    \]

    \item \textbf{Среднеквадратичная скорость:}
    \[
    \langle v^2 \rangle = \int_0^\infty v^2 f(v) dv = 4\pi \left(\dfrac{m}{2\pi k_B T}\right)^{3/2} \int_0^\infty v^4 e^{-mv^2/(2k_B T)} dv
    \]

    Аналогично, используя табличный интеграл:
    \[
    \int_0^\infty v^4 e^{-mv^2/(2k_B T)} dv = \dfrac{3}{4}\sqrt{\pi} \left(\dfrac{2k_B T}{m}\right)^{5/2}
    \]

    Откуда:
    \[
    \langle v^2 \rangle = \dfrac{3k_B T}{m} \quad \Rightarrow \quad v_{rms} = \sqrt{\langle v^2 \rangle} = \sqrt{\dfrac{3k_B T}{m}}
    \]
\end{enumerate}Соотношение между ними:
\[
v_{prob} : \langle v \rangle : v_{rms} = \sqrt{2} : \sqrt{\dfrac{8}{\pi}} : \sqrt{3} \approx 1.41 : 1.60 : 1.73
\]

\textbf{График распределения Максвелла}

\begin{figure}[h]
\centering
\begin{tikzpicture}
\begin{axis}[
    width=0.8\textwidth,
    height=6cm,
    xlabel={Скорость $v/v_{prob}$},
    ylabel={$f(v)$ (отн. ед.)},
    grid=major,
    legend pos=north east,
    domain=0:3,
    samples=200,
    ymin=0,
]
% Безразмерное распределение Максвелла: f(x) = 4/sqrt(pi) * x^2 * exp(-x^2)
% где x = v/v_prob, v_prob = sqrt(2kT/m)
\addplot[blue, thick] {4/sqrt(pi)*x^2*exp(-x^2)};
\addlegendentry{Распределение Максвелла}

% Отметим характерные скорости
\draw[dashed, gray] (axis cs:1,0) -- (axis cs:1,1.2);
\node[anchor=south, gray] at (axis cs:1,1.2) {$v_{prob}$};

\draw[dashed, red] (axis cs:1.128,0) -- (axis cs:1.128,0.8);
\node[anchor=south, red] at (axis cs:1.128,0.8) {$\langle v \rangle$};

\draw[dashed, green!60!black] (axis cs:1.225,0) -- (axis cs:1.225,0.6);
\node[anchor=south, green!60!black] at (axis cs:1.225,0.6) {$v_{rms}$};
\end{axis}
\end{tikzpicture}
\caption{Распределение Максвелла в безразмерных единицах ($v/v_{prob}$). Показаны характерные скорости: наиболее вероятная $v_{prob}$, средняя $\langle v \rangle \approx 1.128 v_{prob}$ и среднеквадратичная $v_{rms} \approx 1.225 v_{prob}$}
\end{figure}

Из графика видно:
\begin{itemize}
    \item Максимум распределения при $v = v_{prob} = \sqrt{2k_BT/m}$
    \item Средняя скорость $\langle v \rangle$ и среднеквадратичная $v_{rms}$ больше наиболее вероятной
    \item Распределение асимметрично -- имеет "хвост" в сторону больших скоростей
\end{itemize}

\textbf{Средняя кинетическая энергия}

\Solution Вычислим среднее значение кинетической энергии:
\[
\left\langle \dfrac{mv^2}{2} \right\rangle = \dfrac{m}{2} \langle v^2 \rangle = \dfrac{m}{2} \cdot \dfrac{3k_B T}{m} = \dfrac{3}{2}k_B T
\]

Альтернативный вывод через интегрирование по компонентам:
\[
\left\langle \dfrac{mv^2}{2} \right\rangle = \left\langle \dfrac{m(v_x^2 + v_y^2 + v_z^2)}{2} \right\rangle = \dfrac{m}{2}(\langle v_x^2 \rangle + \langle v_y^2 \rangle + \langle v_z^2 \rangle)
\]

Для каждой компоненты (например, $v_x$):
\[
\langle v_x^2 \rangle = \int_{-\infty}^{\infty} v_x^2 \sqrt{\dfrac{m}{2\pi k_B T}} e^{-mv_x^2/(2k_B T)} dv_x = \dfrac{k_B T}{m}
\]

Используем табличный интеграл $\int_{-\infty}^{\infty} x^2 e^{-ax^2} dx = \dfrac{1}{2}\sqrt{\dfrac{\pi}{a^3}}$.

Таким образом:
\[
\left\langle \dfrac{mv^2}{2} \right\rangle = \dfrac{m}{2} \cdot 3 \cdot \dfrac{k_B T}{m} = \dfrac{3}{2}k_B T
\]

Это выражение связывает температуру со средней кинетической энергией поступательного движения молекул.

По теореме о равнораспределении энергии по степеням свободы:
\[
\left\langle \dfrac{mv_i^2}{2} \right\rangle = \dfrac{1}{2}k_B T, \quad i = x, y, z
\]

На каждую поступательную степень свободы приходится энергия $\dfrac{1}{2}k_B T$.

\textbf{Проверка нормировки}

\Solution Проверим, что распределение нормировано:
\[
\int_0^\infty f(v) dv = 4\pi \left(\dfrac{m}{2\pi k_B T}\right)^{3/2} \int_0^\infty v^2 e^{-mv^2/(2k_B T)} dv
\]

Используем табличный интеграл (гауссов интеграл):
\[
\int_0^\infty v^2 e^{-av^2} dv = \dfrac{1}{4}\sqrt{\dfrac{\pi}{a^3}}
\]

Где $a = \dfrac{m}{2k_B T}$:
\[
\int_0^\infty v^2 e^{-mv^2/(2k_B T)} dv = \dfrac{1}{4}\sqrt{\dfrac{\pi}{(m/(2k_B T))^3}} = \dfrac{1}{4}\sqrt{\pi} \left(\dfrac{2k_B T}{m}\right)^{3/2}
\]

Подставляем:
\[
\int_0^\infty f(v) dv = 4\pi \left(\dfrac{m}{2\pi k_B T}\right)^{3/2} \cdot \dfrac{1}{4}\sqrt{\pi} \left(\dfrac{2k_B T}{m}\right)^{3/2} = \sqrt{\pi} \cdot \dfrac{1}{\sqrt{\pi}} = 1
\]

Это означает, что вероятность найти молекулу с любой скоростью равна единице.
