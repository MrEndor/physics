\section{Заключение}

В работе проведено комплексное исследование процесса установления термодинамического равновесия в системе идеального газа, находящегося в вертикальном сосуде в гравитационном поле.

\textbf{Основные достижения:}

\begin{enumerate}
    \item \textbf{Численное моделирование динамики молекул}
    \begin{itemize}
        \item Реализована корректная обработка упругих столкновений молекул между собой и со стенками сосуда
        \item Использована оптимизация методом ячеек для ускорения вычислений
    \end{itemize}

    \item \textbf{Исследование процесса релаксации}
    \begin{itemize}
        \item Изучен переход системы из неравновесного начального состояния (все молекулы у дна с одинаковой энергией) к термодинамическому равновесию
        \item Определено характерное время релаксации и его зависимость от параметров системы
        \item Проверено сохранение полной энергии системы в изолированном режиме
    \end{itemize}

    \item \textbf{Получение равновесных распределений}
    \begin{itemize}
        \item Построено распределение молекул по модулям скоростей и проведено сравнение с распределением Максвелла
        \item Построено распределение молекул по высоте и проведено сравнение с барометрической формулой
        \item Проверена факторизация распределений по различным компонентам скорости
    \end{itemize}

    \item \textbf{Статистический анализ}
    \begin{itemize}
        \item Вычислены характерные скорости: наиболее вероятная, средняя и среднеквадратичная
        \item Определена температура системы из распределения кинетических энергий
        \item Проверено выполнение теоремы о равнораспределении энергии по степеням свободы
    \end{itemize}

\end{enumerate}

\textbf{Выводы}

\begin{itemize}
    \item Численное моделирование методом молекулярной динамики подтверждает основные положения статистической механики

    \item Система молекул идеального газа самопроизвольно переходит из упорядоченного неравновесного состояния в равновесное состояние с максимальной энтропией

    \item В равновесии наблюдается полное согласие с распределением Максвелла-Больцмана, несмотря на детерминированный характер классической механики

    
\end{itemize}

