\section{Численные методы решения}

Численное моделирование системы молекул идеального газа основано на методе \textbf{молекулярной динамики} -- пошаговом интегрировании уравнений движения всех частиц с учётом столкновений.

\subsection{Алгоритм интегрирования}

Для интегрирования уравнений движения используется \textbf{метод Верле} (Verlet algorithm), который хорошо сохраняет энергию системы при длительном моделировании.

\textbf{Базовый алгоритм Верле:}

Положение частицы на следующем шаге вычисляется по формуле:
\[
\vec{r}(t + \Delta t) = 2\vec{r}(t) - \vec{r}(t - \Delta t) + \vec{a}(t) \Delta t^2
\]

где $\vec{a}(t) = \dfrac{\vec{F}(t)}{m}$ -- ускорение.

\textbf{Скоростная форма Верле (Velocity Verlet):}

Более удобная модификация, которая явно вычисляет скорости:
\[
\begin{cases}
\vec{r}(t + \Delta t) = \vec{r}(t) + \vec{v}(t) \Delta t + \dfrac{1}{2}\vec{a}(t) \Delta t^2 \\[1em]
\vec{v}(t + \Delta t) = \vec{v}(t) + \dfrac{1}{2}[\vec{a}(t) + \vec{a}(t + \Delta t)] \Delta t
\end{cases}
\]

Для нашей задачи $\vec{a} = \vec{g} = (0, 0, -g) = const$, поэтому:
\[
\begin{cases}
\vec{r}(t + \Delta t) = \vec{r}(t) + \vec{v}(t) \Delta t + \dfrac{1}{2}\vec{g} \Delta t^2 \\[1em]
\vec{v}(t + \Delta t) = \vec{v}(t) + \vec{g} \Delta t
\end{cases}
\]

\textbf{Порядок точности:} метод имеет второй порядок точности по времени, $O(\Delta t^2)$.

\subsection{Обработка столкновений}

\textbf{Столкновения со стенками}

На каждом временном шаге проверяются граничные условия:
\begin{itemize}
    \item \textbf{Дно и крышка:} если $z < 0$ или $z > H$, то:
    \[
    v_z \to -v_z, \quad z \to \max(0, \min(z, H))
    \]

    \item \textbf{Боковая поверхность:} если $\sqrt{x^2 + y^2} > R$, то компонента скорости вдоль радиуса меняет знак:
    \[
    \vec{v}_r \to -\vec{v}_r
    \]
\end{itemize}

Для режима термостата при столкновении со стенкой скорость пересчитывается по распределению Максвелла:
\[
v_i \sim \mathcal{N}\left(0, \sqrt{\dfrac{k_B T}{m}}\right), \quad i = x, y, z
\]

\textbf{Столкновения между молекулами}

Используется \textbf{метод жёстких сфер} (hard sphere model):
\begin{enumerate}
    \item На каждом шаге проверяются расстояния между всеми парами молекул
    \item Если $|\vec{r}_i - \vec{r}_j| < d$ (эффективный диаметр), происходит столкновение
    \item Скорости пересчитываются по формулам упругого столкновения (см. раздел 3.1)
    \item Для избежания повторных столкновений частицы слегка разводятся:
    \[
    \vec{r}_i \to \vec{r}_i + \epsilon \vec{n}, \quad \vec{r}_j \to \vec{r}_j - \epsilon \vec{n}
    \]
    где $\vec{n}$ -- единичный вектор вдоль линии центров, $\epsilon$ -- малая величина
\end{enumerate}

\textbf{Оптимизация вычислений}

Для ускорения проверки столкновений используется метод \textbf{ячеек} (cell lists):
\begin{itemize}
    \item Пространство разбивается на кубические ячейки размером $\ge d$
    \item Каждая молекула приписывается к ячейке по своим координатам
    \item Проверка столкновений производится только с молекулами из той же и соседних ячеек
    \item Сложность алгоритма снижается с $O(N^2)$ до $O(N)$
\end{itemize}

\subsection{Сбор статистики}

После достижения равновесия (время релаксации $t_{relax}$) начинается сбор статистических данных:

\textbf{Распределение по скоростям}

Строится гистограмма модулей скоростей $v = |\vec{v}|$:
\[
n_v(v) = \text{число молекул со скоростями в интервале } [v, v + \Delta v]
\]

Нормированная плотность вероятности:
\[
f_{num}(v) = \dfrac{n_v(v)}{N \cdot \Delta v}
\]

Сравнивается с теоретическим распределением Максвелла.

\textbf{Распределение по высоте}

Строится гистограмма положений по оси $z$:
\[
n_z(z) = \text{число молекул на высотах в интервале } [z, z + \Delta z]
\]

Нормированная плотность вероятности:
\[
f_{num}(z) = \dfrac{n_z(z)}{N \cdot \Delta z}
\]

Сравнивается с барометрической формулой.

\textbf{Критерии равновесия}

Система считается достигшей равновесия, когда:
\begin{enumerate}
    \item Средняя кинетическая энергия флуктуирует около постоянного значения
    \item Распределения по скоростям и высоте не изменяются со временем
    \item Выполняется теорема вириала: $\langle E_k \rangle = \dfrac{3}{2}Nk_B T$
\end{enumerate}

\textbf{Определение температуры}

Температура системы определяется из средней кинетической энергии:
\[
T = \dfrac{2\langle E_k \rangle}{3Nk_B} = \dfrac{2}{3Nk_B} \sum_{i=1}^{N} \dfrac{mv_i^2}{2}
\]

Для проверки можно также использовать распределение по компонентам скорости:
\[
T_i = \dfrac{\langle mv_i^2 \rangle}{k_B}, \quad i = x, y, z
\]

В равновесии $T_x = T_y = T_z = T$.

\newpage
