\begin{center}
    \section{Заключение}
\end{center}

В работе проведено комплексное численное моделирование движения тела, брошенного под углом к горизонту, для трёх различных физических моделей:
\begin{enumerate}
    \item \textbf{Идеальный случай} (без сопротивления воздуха) - аналитически решаемая модель, служащая эталоном для сравнения
    \item \textbf{Модель с вязким трением} ($F \sim v$) - учитывает линейную зависимость силы сопротивления от скорости
    \item \textbf{Модель с лобовым сопротивлением} ($F \sim v^2$) - наиболее реалистичная модель для умеренных и высоких скоростей
\end{enumerate}

\textbf{Основные достижения работы:}

\begin{itemize}
    \item Разработан универсальный алгоритм на основе метода Рунге-Кутты 4-го порядка для численного интегрирования систем дифференциальных уравнений
    \item Проведён детальный аналитический вывод уравнений движения для идеального случая и модели с вязким трением
    \item Получено уравнение траектории $y(x)$ для модели с вязким трением и проведена проверка сходимости к идеальному случаю
    \item Исследовано влияние основных параметров: угла броска $\alpha$, начальной скорости $v_0$, коэффициентов сопротивления $\gamma$ и $\beta$
    \item Обнаружены и проанализированы качественные особенности траекторий для различных моделей
\end{itemize}
