\section{Численные методы решения}

\begin{center}
    \textbf{Преобразование к системе первого порядка}
\end{center}

Исходная система уравнений второго порядка для поступательного и вращательного движения:
\[
\begin{cases}
m\dfrac{d^2\vec{r}}{dt^2} = \vec{F} \\[1em]
I\dfrac{d\vec{\omega}}{dt} = \vec{M}
\end{cases}
\]

Вводим переменные состояния:
\[
\vec{v} = \dfrac{d\vec{r}}{dt}
\]

Получаем систему первого порядка:
\[
\begin{cases}
\dfrac{d\vec{r}}{dt} = \vec{v} \\[1em]
\dfrac{d\vec{v}}{dt} = \dfrac{\vec{F}}{m} \\[1em]
\dfrac{d\vec{\omega}}{dt} = \dfrac{\vec{M}}{I}
\end{cases}
\]

Для движения в плоскости $xOy$ с вращением вокруг оси $z$ перпендикулярной плоскости:
\[
\begin{cases}
\dfrac{dx}{dt} = v_x \\[1em]
\dfrac{dy}{dt} = v_y \\[1em]
\dfrac{dv_x}{dt} = \dfrac{F_x}{m} \\[1em]
\dfrac{dv_y}{dt} = \dfrac{F_y}{m} \\[1em]
\dfrac{d\omega_z}{dt} = \dfrac{M_z}{I_z}
\end{cases}
\]

\begin{center}
    \textbf{Вычисление сил и моментов}
\end{center}

\textbf{1. Для наклонной плоскости (угол $\theta$):}

Компоненты силы тяжести:
\[
F_{g,x} = mg\sin\theta, \quad F_{g,y} = -mg\cos\theta
\]

Нормальная реакция: $N = mg\cos\theta$

Относительная скорость точки контакта:
\[
v_{\text{отн}} = v_x - \omega_z R
\]

Сила трения:
\[
F_{\text{тр}} = \begin{cases}
-\mu N \cdot \sign(v_{\text{отн}}), & \text{если } |v_{\text{отн}}| > \varepsilon \\
F_{\text{тр,покоя}}, & \text{если } |v_{\text{отн}}| \le \varepsilon
\end{cases}
\]

где $\varepsilon$ -- малая пороговая величина для детекции качения без проскальзывания.

При качении без проскальзывания:
\[
F_{\text{тр,покоя}} = \dfrac{2mg\sin\theta}{7}
\]

(ограничена условием $|F_{\text{тр,покоя}}| \le \mu N$)

Момент силы трения:
\[
M_z = F_{\text{тр}} \cdot R
\]

Полные силы:
\[
F_x = mg\sin\theta - F_{\text{тр}}, \quad F_y = 0
\]

\textbf{2. Для горизонтальной плоскости:}

Нормальная реакция: $N = mg$

Относительная скорость точки контакта в векторной форме:
\[
\vec{v}_{\text{отн}} = \vec{v} - \vec{\omega} \times \vec{R}
\]

Для плоского случая ($\omega = \omega_z \vec{e}_z$, $\vec{R} = -R\vec{e}_z$ -- вектор от центра к точке контакта):
\[
\vec{v}_{\text{отн}} = (v_x, v_y) - \omega_z \vec{e}_z \times (-R\vec{e}_z) = (v_x, v_y)
\]

Для учёта вращения в плоскости нужно рассматривать компоненты вращения $\omega_x, \omega_y$. В упрощённой модели:
\[
\vec{v}_{\text{отн}} = (v_x - \omega_y R, v_y + \omega_x R)
\]

Сила трения:
\[
\vec{F}_{\text{тр}} = -\mu mg \dfrac{\vec{v}_{\text{отн}}}{|\vec{v}_{\text{отн}}|}
\]

при $|\vec{v}_{\text{отн}}| > \varepsilon$.

\newpage
\begin{center}
    \textbf{Метод Рунге-Кутты 4-го порядка}
\end{center}

Для системы $\dfrac{d\vec{y}}{dt} = \vec{f}(t, \vec{y})$, где $\vec{y} = (x, y, v_x, v_y, \omega_z)^T$:

\begin{enumerate}
    \item Выбираем шаг интегрирования $h$
    \item На каждом шаге вычисляем коэффициенты:
    \begin{align}
    \vec{k}_1 &= h \vec{f}(t_n, \vec{y}_n) \\
    \vec{k}_2 &= h \vec{f}(t_n + h/2, \vec{y}_n + \vec{k}_1/2) \\
    \vec{k}_3 &= h \vec{f}(t_n + h/2, \vec{y}_n + \vec{k}_2/2) \\
    \vec{k}_4 &= h \vec{f}(t_n + h, \vec{y}_n + \vec{k}_3)
    \end{align}
    \item Обновляем решение:
    \[
    \vec{y}_{n+1} = \vec{y}_n + \dfrac{1}{6}(\vec{k}_1 + 2\vec{k}_2 + 2\vec{k}_3 + \vec{k}_4)
    \]
    \item Обновляем время: $t_{n+1} = t_n + h$
\end{enumerate}

\begin{center}
    \textbf{Алгоритм обработки режимов движения}
\end{center}

На каждом шаге интегрирования:

\begin{enumerate}
    \item Вычисляем относительную скорость: $v_{\text{отн}} = v - \omega R$

    \item Проверяем условие проскальзывания:
    \[
    \text{если } |v_{\text{отн}}| > \varepsilon \text{, то режим проскальзывания}
    \]

    \item \textbf{В режиме проскальзывания:}
    \begin{itemize}
        \item Сила трения: $F_{\text{тр}} = -\mu N \cdot \sign(v_{\text{отн}})$
        \item Интегрируем независимо $v$ и $\omega$
    \end{itemize}

    \item \textbf{В режиме качения без проскальзывания:}
    \begin{itemize}
        \item Вычисляем силу трения из условия связи $a = \alpha R$:
        \[
        F_{\text{тр}} = \dfrac{2ma}{5}
        \]
        где $a$ определяется из уравнения движения с учётом связи
        \item Проверяем условие $|F_{\text{тр}}| \le \mu N$
        \item Если нарушается -- переходим в режим проскальзывания
    \end{itemize}

    \item После интегрирования проверяем переход:
    \[
    \text{если } |v_{n+1} - \omega_{n+1} R| < \varepsilon \text{ и } |F_{\text{тр}}| < \mu N
    \]
    то переходим в режим качения без проскальзывания
\end{enumerate}

\newpage
