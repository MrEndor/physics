\newpage

\section{Введение}

\Task Моделирование движения шара (мяча) по шероховатой поверхности (горизонтальной или наклонной) с учётом вращательного и поступательного движения, сухого трения и возможности проскальзывания. Исследование режимов качения без проскальзывания и с проскальзыванием, а также упругих столкновений с бортами и другими шарами.

\Goal составить программу для численного решения связанной системы дифференциальных уравнений поступательного и вращательного движения шара на шероховатой поверхности, исследовать условия перехода между режимами качения и проскальзывания, проверить выполнение законов сохранения энергии и момента импульса.

\section{Физическая постановка задачи}

Рассмотрим однородный шар радиуса $R$ и массы $m$, движущийся по шероховатой поверхности. На шар действуют следующие силы:

\begin{enumerate}
    \item \textbf{Сила тяжести:} $\vec{F}_g = m\vec{g}$, где $\vec{g}$ -- ускорение свободного падения
    \item \textbf{Сила нормальной реакции опоры:} $\vec{N}$ -- перпендикулярна поверхности
    \item \textbf{Сила трения:} $\vec{F}_{\text{тр}}$ -- действует в точке контакта с поверхностью
\end{enumerate}

\textbf{Координатная система:}

Для наклонной плоскости с углом наклона $\theta$ вводим оси:
\begin{itemize}
    \item Ось $x$ направлена вдоль наклонной плоскости вниз
    \item Ось $y$ перпендикулярна плоскости (направление нормали)
    \item Начало координат в точке, откуда начинается движение
\end{itemize}

Для горизонтальной поверхности используем стандартную декартову систему координат $xOy$.

\textbf{Параметры шара:}
\begin{itemize}
    \item Масса: $m$
    \item Радиус: $R$
    \item Момент инерции относительно центра: $I = \dfrac{2}{5}mR^2$ (для однородного шара)
\end{itemize}

\textbf{Параметры поверхности:}
\begin{itemize}
    \item Угол наклона: $\theta$ (для наклонной плоскости)
    \item Коэффициент сухого трения: $\mu$
\end{itemize}

\textbf{Режимы движения:}
\begin{enumerate}
    \item \textbf{Качение без проскальзывания:} связь между линейной и угловой скоростями $v = \omega R$
    \item \textbf{Качение с проскальзыванием:} условие связи нарушается, $v \neq \omega R$
    \item \textbf{Чистое скольжение:} вращение отсутствует или пренебрежимо мало, $\omega \approx 0$
\end{enumerate}

\textbf{Начальные условия:}

В общем случае при $t = 0$:
\[
\vec{r}(0) = \vec{r}_0, \quad \vec{v}(0) = \vec{v}_0, \quad \vec{\omega}(0) = \vec{\omega}_0
\]

где $\vec{v}_0$ -- начальная линейная скорость центра масс, $\vec{\omega}_0$ -- начальная угловая скорость.

\newpage
