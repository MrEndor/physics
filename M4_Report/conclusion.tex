\section{Заключение}

В работе проведено комплексное исследование динамики шара, катящегося по шероховатой поверхности, с учётом поступательного и вращательного движения, трения и возможности перехода между различными режимами движения.

\textbf{Основные выводы:}

\begin{itemize}
    \item При качении шара без проскальзывания по наклонной плоскости ускорение составляет 
    
    $a = \dfrac{5g\sin\theta}{7}$, что меньше ускорения при чистом скольжении ($a = g\sin\theta$) за счёт затрат энергии на вращение

    \item Условие качения без проскальзывания: $\tg\theta \le \dfrac{7\mu}{2}$. 
    
    При нарушении этого условия происходит переход в режим проскальзывания

    \item Установившаяся скорость при качении без проскальзывания определяется законами сохранения импульса и момента импульса: $v_{\text{уст}} = \dfrac{5v_0 + 2\omega_0 R}{7}$

    \item Диссипация энергии при установлении качения без проскальзывания полностью определяется работой силы трения скольжения: $Q = \int F_{\text{тр}} \cdot |v_{\text{отн}}| dt$
\end{itemize}

