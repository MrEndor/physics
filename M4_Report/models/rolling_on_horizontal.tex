\subsection{Произвольное качение по горизонтальной плоскости}

Рассмотрим шар, движущийся по горизонтальной шероховатой поверхности с произвольными начальными линейной $\vec{v}_0$ и угловой $\vec{\omega}_0$ скоростями.

\begin{center}
    \textbf{Уравнения движения в векторной форме}
\end{center}

Для горизонтальной плоскости $\vec{g}$ направлено вертикально вниз, нормальная реакция уравновешивает силу тяжести: $N = mg$.

\textbf{Поступательное движение:}
\[
m\dfrac{d\vec{v}}{dt} = \vec{F}_{\text{тр}}
\]

\textbf{Вращательное движение:}
\[
I\dfrac{d\vec{\omega}}{dt} = \vec{M}
\]

где $\vec{M} = \vec{R} \times \vec{F}_{\text{тр}}$ -- момент силы трения.

Для плоского движения (вращение вокруг вертикальной оси $z$, перпендикулярной плоскости):
\[
I_z \dfrac{d\omega_z}{dt} = M_z = R \cdot F_{\text{тр}}
\]

\begin{center}
    \textbf{Сила трения при произвольном движении}
\end{center}

Относительная скорость точки контакта шара с поверхностью:
\[
\vec{v}_{\text{отн}} = \vec{v}_{\text{центра}} - \vec{v}_{\text{вращ}} = \vec{v} - \vec{\omega} \times \vec{R}
\]

Для вращения вокруг вертикальной оси и движения в горизонтальной плоскости $xOy$:
\[
\vec{v}_{\text{отн}} = (v_x, v_y) - (-\omega_z R \sin\varphi, \omega_z R \cos\varphi)
\]

где $\varphi$ -- угол между направлением движения и осью $x$.

В общем случае в декартовых координатах:
\[
\vec{v}_{\text{отн}} = (v_x - \omega_y R, v_y + \omega_x R)
\]

\textbf{Сила трения скольжения:}
\[
\vec{F}_{\text{тр}} = -\mu N \dfrac{\vec{v}_{\text{отн}}}{|\vec{v}_{\text{отн}}|} = -\mu mg \dfrac{\vec{v}_{\text{отн}}}{|\vec{v}_{\text{отн}}|}
\]

при $|\vec{v}_{\text{отн}}| > 0$ (проскальзывание).

\newpage
\begin{center}
    \textbf{Случай 1: Движение вдоль одной оси}
\end{center}

Рассмотрим движение вдоль оси $x$ с начальными условиями:
\[
v_x(0) = v_0, \quad v_y(0) = 0, \quad \omega_z(0) = \omega_0
\]

Относительная скорость точки контакта:
\[
v_{\text{отн}} = v_x - \omega_z R
\]

\textbf{При проскальзывании} ($v_{\text{отн}} \neq 0$):
\[
\begin{cases}
\dfrac{dv_x}{dt} = -\mu g \cdot \sign(v_{\text{отн}}) \\[1em]
\dfrac{d\omega_z}{dt} = \dfrac{\mu mg R}{I_z} \cdot \sign(v_{\text{отн}}) = \dfrac{5\mu g}{2R} \cdot \sign(v_{\text{отн}})
\end{cases}
\]

\textbf{Анализ эволюции системы:}

\textit{Случай A:} $v_0 > \omega_0 R$ (центр движется быстрее, чем "требует" вращение)
\[
v_{\text{отн}} > 0 \quad \Rightarrow \quad \sign(v_{\text{отн}}) = +1
\]
\[
\begin{cases}
\dfrac{dv_x}{dt} = -\mu g \quad &\text{(линейная скорость уменьшается)} \\[1em]
\dfrac{d\omega_z}{dt} = +\dfrac{5\mu g}{2R} \quad &\text{(угловая скорость увеличивается)}
\end{cases}
\]

Со временем $v_x$ уменьшается, а $\omega_z$ растёт, пока не выполнится $v_x = \omega_z R$.

\textit{Случай B:} $v_0 < \omega_0 R$ (вращение "опережает" поступательное движение)
\[
v_{\text{отн}} < 0 \quad \Rightarrow \quad \sign(v_{\text{отн}}) = -1
\]
\[
\begin{cases}
\dfrac{dv_x}{dt} = +\mu g \quad &\text{(линейная скорость увеличивается)} \\[1em]
\dfrac{d\omega_z}{dt} = -\dfrac{5\mu g}{2R} \quad &\text{(угловая скорость уменьшается)}
\end{cases}
\]

Система эволюционирует к состоянию $v_x = \omega_z R$.

\textbf{Время установления качения без проскальзывания:}

Пусть $v_0 > \omega_0 R$. Тогда:
\[
\begin{aligned}
v_x(t) &= v_0 - \mu g t \\
\omega_z(t) &= \omega_0 + \dfrac{5\mu g}{2R} t
\end{aligned}
\]

Условие $v_x(t^*) = \omega_z(t^*) R$:
\[
v_0 - \mu g t^* = \left(\omega_0 + \dfrac{5\mu g}{2R} t^*\right) R
\]
\[
v_0 - \mu g t^* = \omega_0 R + \dfrac{5\mu g}{2} t^*
\]
\[
v_0 - \omega_0 R = \mu g t^* + \dfrac{5\mu g}{2} t^* = \dfrac{7\mu g}{2} t^*
\]
\[
\boxed{t^* = \dfrac{2(v_0 - \omega_0 R)}{7\mu g}}
\]

\textbf{После установления качения без проскальзывания:}

Сила трения становится силой трения покоя и не совершает работу. Шар движется равномерно с постоянной скоростью:
\[
v_{\text{уст}} = \omega_{\text{уст}} R
\]

где
\[
v_{\text{уст}} = v_0 - \mu g t^* = v_0 - \dfrac{2(v_0 - \omega_0 R)}{7}
\]
\[
\boxed{v_{\text{уст}} = \dfrac{5v_0 + 2\omega_0 R}{7}}
\]

\begin{center}
    \textbf{Случай 2: Движение по окружности}
\end{center}

Рассмотрим шар, запущенный по горизонтальной плоскости так, что его центр движется по окружности радиуса $r$ с начальной линейной скоростью $v_0$ и угловой скоростью вращения $\omega_0$.

Для движения по окружности требуется центростремительное ускорение:
\[
a_n = \dfrac{v^2}{r}
\]

которое обеспечивается силой трения:
\[
F_{\text{тр}} = m\dfrac{v^2}{r}
\]

При этом должно выполняться:
\[
F_{\text{тр}} \le \mu N = \mu mg
\]

Условие возможности движения по окружности:
\[
\dfrac{v^2}{r} \le \mu g \quad \Rightarrow \quad \boxed{v \le \sqrt{\mu g r}}
\]

При превышении этой скорости шар соскользнёт с траектории окружности.

\begin{center}
    \textbf{Энергетические соотношения}
\end{center}

При движении по горизонтальной плоскости с трением механическая энергия уменьшается:
\[
\dfrac{dE}{dt} = -F_{\text{тр}} \cdot v_{\text{отн}} = -\mu mg |\vec{v}_{\text{отн}}|
\]

Полная энергия системы:
\[
E = \dfrac{1}{2}mv^2 + \dfrac{1}{2}I\omega^2 = \dfrac{1}{2}mv^2 + \dfrac{1}{5}mR^2\omega^2
\]

После установления режима качения без проскальзывания ($v = \omega R$):
\[
E_{\text{уст}} = \dfrac{7}{10}mv_{\text{уст}}^2
\]

Потерянная энергия:
\[
\Delta E = E_0 - E_{\text{уст}} = \dfrac{1}{2}mv_0^2 + \dfrac{1}{5}mR^2\omega_0^2 - \dfrac{7}{10}mv_{\text{уст}}^2
\]
