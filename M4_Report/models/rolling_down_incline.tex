\subsection{Скатывание по наклонной плоскости}

Рассмотрим шар, скатывающийся по наклонной плоскости с углом наклона $\theta$ к горизонту.

\textbf{Уравнения движения}

Применяем \textbf{второй закон Ньютона} для поступательного движения центра масс:
\[
m\vec{a} = \vec{F}_g + \vec{N} + \vec{F}_{\text{тр}}
\]

В проекциях на оси (ось $x$ вдоль плоскости вниз, $y$ перпендикулярно плоскости):
\[
\begin{cases}
ma_x = mg\sin\theta - F_{\text{тр}} \\
ma_y = N - mg\cos\theta = 0
\end{cases}
\]

Применяем \textbf{основное уравнение динамики вращательного движения}:
\[
I\dfrac{d\omega}{dt} = M
\]

где $M$ -- момент силы трения относительно центра масс:
\[
M = F_{\text{тр}} \cdot R
\]

Для однородного шара $I = \dfrac{2}{5}mR^2$, следовательно:
\[
\dfrac{2}{5}mR^2 \dfrac{d\omega}{dt} = F_{\text{тр}} R
\]
\[
\dfrac{d\omega}{dt} = \dfrac{5F_{\text{тр}}}{2mR}
\]

\begin{center}
    \textbf{Качение без проскальзывания}
\end{center}

При качении без проскальзывания выполняется \textbf{условие связи}:
\[
v = \omega R \quad \Rightarrow \quad a = \dfrac{dv}{dt} = R\dfrac{d\omega}{dt}
\]

Сила трения в этом режиме -- \textbf{сила трения покоя} $F_{\text{тр}} \le \mu N = \mu mg\cos\theta$.

Подставляем условие связи в уравнение вращения:
\[
\dfrac{a}{R} = \dfrac{5F_{\text{тр}}}{2mR} \quad \Rightarrow \quad F_{\text{тр}} = \dfrac{2ma}{5}
\]

Подставляем в уравнение поступательного движения:
\[
ma = mg\sin\theta - \dfrac{2ma}{5}
\]
\[
ma + \dfrac{2ma}{5} = mg\sin\theta
\]
\[
\dfrac{7ma}{5} = mg\sin\theta
\]

Отсюда ускорение центра масс при качении без проскальзывания:
\[
\boxed{a = \dfrac{5g\sin\theta}{7}}
\]

Соответствующая сила трения:
\[
F_{\text{тр}} = \dfrac{2ma}{5} = \dfrac{2mg\sin\theta}{7}
\]

\textbf{Условие качения без проскальзывания:}
\[
F_{\text{тр}} \le \mu N \quad \Rightarrow \quad \dfrac{2mg\sin\theta}{7} \le \mu mg\cos\theta
\]
\[
\boxed{\tg\theta \le \dfrac{7\mu}{2}}
\]

При нарушении этого условия происходит переход в режим проскальзывания.

\begin{center}
    \textbf{Качение с проскальзыванием}
\end{center}

При проскальзывании условие связи $v = \omega R$ нарушается, и действует \textbf{сила трения скольжения}:
\[
F_{\text{тр}} = \mu N = \mu mg\cos\theta
\]

Направление силы трения определяется знаком относительной скорости точки контакта:
\[
v_{\text{отн}} = v - \omega R
\]
\[
F_{\text{тр}} = -\mu mg\cos\theta \cdot \sign(v_{\text{отн}})
\]

Система уравнений в режиме проскальзывания:
\[
\begin{cases}
\dfrac{dv}{dt} = g\sin\theta - \mu g\cos\theta \cdot \sign(v - \omega R) \\[1em]
\dfrac{d\omega}{dt} = \dfrac{5\mu g\cos\theta}{2R} \cdot \sign(v - \omega R)
\end{cases}
\]

\textbf{Переход в режим качения без проскальзывания:}

Происходит при выполнении условий:
\begin{enumerate}
    \item $|v - \omega R| < \varepsilon$ (малая относительная скорость)
    \item $\left|\dfrac{2mg\sin\theta}{7}\right| < \mu mg\cos\theta$ (условие для силы трения покоя)
\end{enumerate}

\begin{center}
    \textbf{Аналитическое решение для качения без проскальзывания}
\end{center}

При $a = \text{const} = \dfrac{5g\sin\theta}{7}$ и начальных условиях $v(0) = 0$, $x(0) = 0$:
\[
v(t) = \dfrac{5g\sin\theta}{7} t
\]
\[
x(t) = \dfrac{1}{2} \cdot \dfrac{5g\sin\theta}{7} t^2 = \dfrac{5g\sin\theta}{14} t^2
\]

Угловая скорость:
\[
\omega(t) = \dfrac{v(t)}{R} = \dfrac{5g\sin\theta}{7R} t
\]

\textbf{Анализ энергий:}

Полная механическая энергия при скатывании без проскальзывания:
\[
E = E_{\text{кин}} + E_{\text{пот}} = \dfrac{1}{2}mv^2 + \dfrac{1}{2}I\omega^2 + mgh
\]

При $v = \omega R$ и $I = \dfrac{2}{5}mR^2$:
\[
E_{\text{кин}} = \dfrac{1}{2}mv^2 + \dfrac{1}{2} \cdot \dfrac{2}{5}mR^2 \cdot \dfrac{v^2}{R^2} = \dfrac{1}{2}mv^2 + \dfrac{1}{5}mv^2 = \dfrac{7}{10}mv^2
\]

При скатывании с высоты $h = x\sin\theta$ из состояния покоя:
\[
mgh = \dfrac{7}{10}mv^2 \quad \Rightarrow \quad v = \sqrt{\dfrac{10gh}{7}}
\]

Это на $\sqrt{\dfrac{10}{7}} \approx 1.195$ раза меньше скорости при чистом скольжении ($v = \sqrt{2gh}$), что объясняется затратами энергии на вращение.
