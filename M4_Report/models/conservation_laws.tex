\subsection{Законы сохранения энергии и момента импульса}

Проверка выполнения законов сохранения является важным критерием корректности численного моделирования.

\begin{center}
    \textbf{Закон сохранения энергии}
\end{center}

\textbf{Полная механическая энергия системы:}
\[
E = E_{\text{кин}} + E_{\text{вращ}} + E_{\text{пот}}
\]

где:
\begin{itemize}
    \item $E_{\text{кин}} = \dfrac{1}{2}mv^2$ -- кинетическая энергия поступательного движения
    \item $E_{\text{вращ}} = \dfrac{1}{2}I\omega^2 = \dfrac{1}{5}mR^2\omega^2$ -- энергия вращательного движения
    \item $E_{\text{пот}} = mgh$ -- потенциальная энергия в поле тяжести
\end{itemize}

\textbf{Для одного шара:}
\[
E = \dfrac{1}{2}mv^2 + \dfrac{1}{5}mR^2\omega^2 + mgh
\]

\textbf{Изменение энергии:}

\textit{1. При качении без проскальзывания по наклонной плоскости:}

Трение покоя не совершает работу ($\vec{F}_{\text{тр}} \perp \vec{v}_{\text{контакта}}$), поэтому энергия сохраняется:
\[
\dfrac{dE}{dt} = 0 \quad \Rightarrow \quad E = \text{const}
\]

\textit{2. При проскальзывании:}

Трение скольжения совершает отрицательную работу:
\[
\dfrac{dE}{dt} = -F_{\text{тр}} \cdot v_{\text{отн}} = -\mu mg\cos\theta \cdot |v - \omega R|
\]

Энергия уменьшается:
\[
E(t) = E_0 - \int_0^t \mu mg\cos\theta \cdot |v(\tau) - \omega(\tau) R| d\tau
\]

\textit{3. При упругих столкновениях:}

В идеальном случае упругого столкновения энергия сохраняется:
\[
E_{\text{до}} = E_{\text{после}}
\]

При использовании модели с силой Гука энергия сохраняется, если пренебречь диссипацией.

\begin{center}
    \textbf{Закон сохранения момента импульса}
\end{center}

\textbf{Момент импульса материальной точки} относительно точки $O$:
\[
\vec{L}_{\text{орб}} = \vec{r} \times m\vec{v}
\]

\textbf{Собственный момент импульса} вращающегося тела:
\[
\vec{L}_{\text{собств}} = I\vec{\omega}
\]

\textbf{Полный момент импульса шара:}
\[
\vec{L} = \vec{r} \times m\vec{v} + I\vec{\omega}
\]

Для движения в плоскости $xOy$ с вращением вокруг оси $z$:
\[
L_z = (x \cdot m v_y - y \cdot m v_x) + I\omega_z
\]

\textbf{Условия сохранения момента импульса:}

Момент импульса сохраняется, если суммарный момент внешних сил равен нулю:
\[
\sum \vec{M}_{\text{внешн}} = 0
\]

\textit{1. Для шара на наклонной плоскости:}

Относительно точки на оси вращения (перпендикулярной плоскости движения) момент силы тяжести и нормальной реакции равны нулю (они проходят через эту ось). Момент силы трения:
\[
M = F_{\text{тр}} \cdot R
\]

не равен нулю, поэтому момент импульса не сохраняется.

\textit{2. Для изолированной системы шаров на горизонтальной плоскости:}

Если внешние силы (тяжесть, нормальная реакция) не создают момента относительно вертикальной оси, и трение о поверхность также не создаёт вертикального момента, то:
\[
\dfrac{d\vec{L}_z}{dt} = 0 \quad \Rightarrow \quad L_z = \text{const}
\]

\textit{3. При упругих столкновениях:}

Если столкновение происходит в изолированной системе, момент импульса сохраняется:
\[
\vec{L}_{\text{до}} = \vec{L}_{\text{после}}
\]

\newpage
\begin{center}
    \textbf{Диссипация энергии при установлении качения}
\end{center}

При переходе от проскальзывания к качению без проскальзывания часть энергии диссипирует.

Начальная энергия (при $v_0$ и $\omega_0$):
\[
E_0 = \dfrac{1}{2}mv_0^2 + \dfrac{1}{5}mR^2\omega_0^2
\]

Конечная энергия (при $v_{\text{уст}} = \omega_{\text{уст}} R$):
\[
E_{\text{уст}} = \dfrac{7}{10}mv_{\text{уст}}^2
\]

Диссипированная энергия:
\[
\Delta E = E_0 - E_{\text{уст}}
\]

Эта энергия переходит в тепло за счёт работы силы трения скольжения:
\[
Q = \int_0^{t^*} F_{\text{тр}} \cdot |v_{\text{отн}}| dt = \int_0^{t^*} \mu mg |v - \omega R| dt
\]

Численная проверка: $Q \approx \Delta E$ (с точностью до ошибок численного интегрирования).
