\begin{center}
    \subsection{Модель с вязким трением ($F \sim v$)}
\end{center}
    
Сила сопротивления пропорциональна скорости:
\[
\vec{F}_{\text{сопр}} = -k\vec{v}
\]
где $k > 0$ -- коэффициент сопротивления.

Второй закон Ньютона:
\[
m\dfrac{d^2\vec{r}}{dt^2} = m\vec{g} - k\vec{v}
\]
или в проекциях:
\[
\begin{cases}
m\dfrac{d^2x}{dt^2} = -k\dfrac{dx}{dt} \\[1em]
m\dfrac{d^2y}{dt^2} = -mg - k\dfrac{dy}{dt}
\end{cases}
\]

Введя обозначения $\gamma = \dfrac{k}{m}$ --- коэффициент затухания, получаем:
\[
\begin{cases}
\dfrac{d^2x}{dt^2} + \gamma \dfrac{dx}{dt} = 0 \\[1em]
\dfrac{d^2y}{dt^2} + \gamma \dfrac{dy}{dt} = -g
\end{cases}
\]

\Solution Решаем систему дифференциальных уравнений.

\textbf{Горизонтальное движение:}
\[
\dfrac{d^2x}{dt^2} + \gamma \dfrac{dx}{dt} = 0
\]

Введем обозначение для скорости: $v_x = \dfrac{dx}{dt}$. Тогда уравнение принимает вид:
\[
\dfrac{dv_x}{dt} + \gamma v_x = 0
\]

Разделяем переменные и интегрируем от 0 до $t$:
\[
\int_{v_0 \cos\alpha}^{v_x(t)} \dfrac{dv_x}{v_x} = -\gamma \int_0^t dt
\]
\[
\ln\left(\dfrac{v_x}{v_0 \cos\alpha}\right) = -\gamma t
\]
\[
v_x(t) = v_0 \cos\alpha \cdot e^{-\gamma t}
\]

Интегрируем для нахождения координаты $x$:
\[
\int_0^{x(t)} dx = \int_0^t v_0 \cos\alpha \cdot e^{-\gamma t} dt
\]
\[
x(t) = v_0 \cos\alpha \left[-\dfrac{e^{-\gamma t}}{\gamma}\right]_0^t = \dfrac{v_0 \cos\alpha}{\gamma}(1 - e^{-\gamma t})
\]

\vspace{0.5cm}
\textbf{Вертикальное движение:}
\[
\dfrac{d^2y}{dt^2} + \gamma \dfrac{dy}{dt} = -g
\]

Введем обозначение для скорости: $v_y = \dfrac{dy}{dt}$. Тогда уравнение принимает вид:
\[
\dfrac{dv_y}{dt} + \gamma v_y = -g
\]

Это линейное неоднородное уравнение первого порядка. Решаем методом вариации постоянной.

Сначала находим общее решение однородного уравнения:
\[
\dfrac{dv_y}{dt} + \gamma v_y = 0
\]
Разделяем переменные:
\[
\int \dfrac{dv_y}{v_y} = -\gamma \int dt
\]
\[
\ln|v_y| = -\gamma t + C
\]
\[
v_y^{\text{одн}} = C e^{-\gamma t}
\]

Теперь ищем решение неоднородного уравнения в виде:
\[
v_y(t) = C(t) e^{-\gamma t}
\]

Подставляем в исходное уравнение:
\[
\dfrac{d}{dt}\left[C(t) e^{-\gamma t}\right] + \gamma C(t) e^{-\gamma t} = -g
\]
\[
C'(t) e^{-\gamma t} - \gamma C(t) e^{-\gamma t} + \gamma C(t) e^{-\gamma t} = -g
\]
\[
C'(t) e^{-\gamma t} = -g
\]
\[
C'(t) = -g e^{\gamma t}
\]

Интегрируем:
\[
C(t) = -g \int e^{\gamma t} dt = -\dfrac{g}{\gamma} e^{\gamma t} + A
\]

Подставляем обратно:
\[
v_y(t) = \left(-\dfrac{g}{\gamma} e^{\gamma t} + A\right) e^{-\gamma t} = -\dfrac{g}{\gamma} + A e^{-\gamma t}
\]

Из начального условия $v_y(0) = v_0 \sin\alpha$:
\[
-\dfrac{g}{\gamma} + A = v_0 \sin\alpha \Rightarrow A = v_0 \sin\alpha + \dfrac{g}{\gamma}
\]

Таким образом:
\[
v_y(t) = \left(v_0 \sin\alpha + \dfrac{g}{\gamma}\right)e^{-\gamma t} - \dfrac{g}{\gamma}
\]

Интегрируем для нахождения координаты $y$:
\[
\int_0^y dy = \int_0^t \left[\left(v_0 \sin\alpha + \dfrac{g}{\gamma}\right)e^{-\gamma t} - \dfrac{g}{\gamma}\right] dt
\]
\[
y(t) = \left(v_0 \sin\alpha + \dfrac{g}{\gamma}\right)\left[-\dfrac{1}{\gamma}e^{-\gamma t}\right]_0^t - \dfrac{g}{\gamma}t
\]
\[
y(t) = -\dfrac{1}{\gamma}\left(v_0 \sin\alpha + \dfrac{g}{\gamma}\right)(e^{-\gamma t} - 1) - \dfrac{g}{\gamma}t
\]
\[
y(t) = \dfrac{1}{\gamma}\left(v_0 \sin\alpha + \dfrac{g}{\gamma}\right)(1 - e^{-\gamma t}) - \dfrac{g}{\gamma}t
\]

\vspace{0.5cm}
\textbf{Вывод уравнения траектории $\bm{y(x)}$:}

Из уравнения для горизонтальной координаты:
\[
x(t) = \dfrac{v_0 \cos\alpha}{\gamma}(1 - e^{-\gamma t})
\]
Выразим время через координату $x$:
\[
1 - e^{-\gamma t} = \dfrac{\gamma x}{v_0 \cos\alpha}
\]
\[
t = -\dfrac{1}{\gamma} \ln\left(1 - \dfrac{\gamma x}{v_0 \cos\alpha}\right)
\]

Теперь подставим эти выражения в уравнение для вертикальной координаты:
\[
y(t) = \dfrac{1}{\gamma}\left(v_0 \sin\alpha + \dfrac{g}{\gamma}\right)(1 - e^{-\gamma t}) - \dfrac{g}{\gamma}t
\]

Получаем:
\[
y(x) = \dfrac{1}{\gamma}\left(v_0 \sin\alpha + \dfrac{g}{\gamma}\right)\cdot \left(\dfrac{\gamma x}{v_0 \cos\alpha}\right) + \dfrac{g}{\gamma^2} \ln\left(1 - \dfrac{\gamma x}{v_0 \cos\alpha}\right)
\]

Упрощаем:
\[
y(x) = x \tan\alpha + \dfrac{g x}{\gamma v_0 \cos\alpha} + \dfrac{g}{\gamma^2} \ln\left(1 - \dfrac{\gamma x}{v_0 \cos\alpha}\right)
\]

Это и есть уравнение траектории для модели с вязким трением.

\textbf{Проверка предела при $\bm{\gamma \to 0}$:}
Используем разложение логарифма в ряд Тейлора:
\[
\ln(1 - u) = -u - \dfrac{u^2}{2} - \dfrac{u^3}{3} - \cdots, \quad \text{где } u = \dfrac{\gamma x}{v_0 \cos\alpha}
\]

Тогда:
\[
y(x) = x \tan\alpha + \dfrac{g x}{\gamma v_0 \cos\alpha} + \dfrac{g}{\gamma^2} \left(-\dfrac{\gamma x}{v_0 \cos\alpha} - \dfrac{\gamma^2 x^2}{2v_0^2 \cos^2\alpha} + o(\gamma^3)\right)
\]
\[
= x \tan\alpha + \dfrac{g x}{\gamma v_0 \cos\alpha} - \dfrac{g x}{\gamma v_0 \cos\alpha} - \dfrac{g x^2}{2v_0^2 \cos^2\alpha} + o(\gamma)
\]
\[
= x \tan\alpha - \dfrac{g x^2}{2v_0^2 \cos^2\alpha} + o(\gamma) \xrightarrow{\gamma \to 0} x \tan\alpha - \dfrac{g x^2}{2v_0^2 \cos^2\alpha}
\]

Что совпадает с уравнением траектории для идеального случая.