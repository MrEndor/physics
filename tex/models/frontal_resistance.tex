\begin{center}
    \subsection{Модель с лобовым сопротивлением ($F \sim v^2$)}
\end{center}

Сила сопротивления пропорциональна квадрату скорости и направлена противоположно вектору скорости:
\[
\vec{F}_{\text{сопр}} = -c|\vec{v}|\vec{v} = -c v \vec{v}
\]
где $c > 0$ -- коэффициент лобового сопротивления, $v = |\vec{v}| = \sqrt{v_x^2 + v_y^2}$.

Второй закон Ньютона:
\[
m\dfrac{d^2\vec{r}}{dt^2} = m\vec{g} - c v \vec{v}
\]
или в проекциях на оси:
\[
\begin{cases}
m\dfrac{d^2x}{dt^2} = -c\sqrt{\left(\dfrac{dx}{dt}\right)^2 + \left(\dfrac{dy}{dt}\right)^2} \cdot \dfrac{dx}{dt} \\[1em]
m\dfrac{d^2y}{dt^2} = -mg - c\sqrt{\left(\dfrac{dx}{dt}\right)^2 + \left(\dfrac{dy}{dt}\right)^2} \cdot \dfrac{dy}{dt}
\end{cases}
\]

Введя обозначение $\beta = \dfrac{c}{m}$, получаем:
\[
\begin{cases}
\dfrac{d^2x}{dt^2} = -\beta v \dfrac{dx}{dt} \\[1em]
\dfrac{d^2y}{dt^2} = -g - \beta v \dfrac{dy}{dt}
\end{cases}
\]
где $v = \sqrt{\left(\dfrac{dx}{dt}\right)^2 + \left(\dfrac{dy}{dt}\right)^2}$.

\Solution Данная система не имеет аналитического решения в элементарных функциях и требует численных методов решения.

\textbf{Приведение к системе первого порядка:}
Введём переменные:
\[
v_x = \dfrac{dx}{dt}, \quad v_y = \dfrac{dy}{dt}
\]
Тогда система принимает вид:
\[
\begin{cases}
\dfrac{dx}{dt} = v_x \\[1em]
\dfrac{dy}{dt} = v_y \\[1em]
\dfrac{dv_x}{dt} = -\beta \sqrt{v_x^2 + v_y^2} \cdot v_x \\[1em]
\dfrac{dv_y}{dt} = -g - \beta \sqrt{v_x^2 + v_y^2} \cdot v_y
\end{cases}
\]

\textbf{Начальные условия:}
\[
x(0) = 0, \quad y(0) = 0, \quad v_x(0) = v_0 \cos\alpha, \quad v_y(0) = v_0 \sin\alpha
\]

\newpage 

\begin{center}
    \textbf{Анализ особенностей:}
\end{center}
    
\begin{enumerate}
    \item Система нелинейна из-за члена $\sqrt{v_x^2 + v_y^2}$
    \item Уравнения связаны через общий множитель $\sqrt{v_x^2 + v_y^2}$
    \item При $\beta = 0$ система переходит в идеальный случай
    \item При малых скоростях ($v \to 0$) система приближается к модели с вязким трением
    \item \textbf{Предельная скорость:} При вертикальном падении ($v_x = 0$) уравнение движения принимает вид:
    \[
    m\dfrac{dv_y}{dt} = -mg - c v_y^2
    \]
    Предельная скорость достигается при условии $\dfrac{dv_y}{dt} = 0$:
    \[
    0 = -mg - c v_y^2 \quad \Rightarrow \quad c v_y^2 = -mg
    \]
    Учитывая, что при падении $v_y < 0$, получаем:
    \[
    v_{\text{предел}} = -\sqrt{\dfrac{mg}{c}} = -\sqrt{\dfrac{g}{\beta}}
    \]
    где $\beta = \dfrac{c}{m}$. Физически это означает, что сила сопротивления $c v^2$ уравновешивает силу тяжести $mg$, и тело движется с постоянной скоростью.
    \item \textbf{Асимметрия траектории:} Траектория становится несимметричной - восходящая ветвь более пологая, чем нисходящая, так как на подъёме сила сопротивления направлена против скорости и силы тяжести, а на спуске - против скорости, но совместно с силой тяжести
    \item \textbf{Зависимость от начальной скорости:} Влияние сопротивления растет пропорционально квадрату скорости, поэтому при больших начальных скоростях эффект сопротивления становится доминирующим
    \item \textbf{Оптимальный угол броска:} Угол, обеспечивающий максимальную дальность, становится меньше $45^\circ$, поскольку горизонтальная компонента скорости затухает быстрее вертикальной из-за квадратичной зависимости силы сопротивления
\end{enumerate}
