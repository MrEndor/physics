\newpage

\begin{center}
    \section{Введение}
\end{center}

\Task Исследование движения камня, брошенного под углом $\alpha$ к горизонту в земном поле тяжести с начальной скоростью $\vec{v}_0$, с учётом различных моделей сопротивления воздуха.

\Goal составить программу для численного решения дифференциального уравнения движения камня, рассчитать его траекторию, определить точку падения и исследовать влияние начальных параметров броска и коэффициентов сопротивления на характер траектории.

\begin{center}
    \section{Физическая постановка задачи}
\end{center}
\vspace{-2em}
Рассмотрим движение материальной точки массы $m$ в поле тяжести Земли. На точку действуют:
\begin{enumerate}
    \item Сила тяжести: $\vec{F}_g = m\vec{g}$, где $\vec{g} = \begin{pmatrix} 0 \\ -g \end{pmatrix}$ и $g \approx 9.81$ м/с$^2$
    \item Сила сопротивления воздуха: $\vec{F}_{\text{сопр}}$, зависящая от модели
\end{enumerate}

Начальные условия при $t = 0$:
\[
\vec{r}(0) = \begin{pmatrix} 0 \\ 0 \end{pmatrix}, \quad \vec{v}(0) = \begin{pmatrix} v_0 \cos\alpha \\ v_0 \sin\alpha \end{pmatrix}
\]
где $\alpha$ -- угол броска относительно горизонта, $v_0$ -- начальная скорость.

\newpage
