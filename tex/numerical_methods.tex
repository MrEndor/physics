\begin{center}
    \section{Численные методы решения}
\end{center}

Для решения системы дифференциальных уравнений второго порядка преобразуем её в систему первого порядка. Введём переменные:
\[
v_x = \frac{dx}{dt}, \quad v_y = \frac{dy}{dt}
\]

Тогда для модели с лобовым сопротивлением получаем систему:
\[
\begin{cases}
\dfrac{dx}{dt} = v_x \\[1em]
\dfrac{dy}{dt} = v_y \\[1em]
\dfrac{dv_x}{dt} = -\beta \sqrt{v_x^2 + v_y^2} \cdot v_x \\[1em]
\dfrac{dv_y}{dt} = -g - \beta \sqrt{v_x^2 + v_y^2} \cdot v_y
\end{cases}
\]

Для численного интегрирования используем метод Рунге-Кутты 4-го порядка. Шаг алгоритма:
\begin{enumerate}
    \item Задаём начальные условия: $x_0, y_0, v_{x0}, v_{y0}$
    \item Выбираем шаг по времени $\Delta t$
    \item На каждом шаге вычисляем коэффициенты $k_1, k_2, k_3, k_4$ для каждой переменной:
    \begin{itemize}
        \item $k_1$ - наклон в начале интервала (аппроксимация методом Эйлера)
        \item $k_2$ - наклон в средней точке с использованием $k_1$
        \item $k_3$ - улучшенный наклон в средней точке с использованием $k_2$  
        \item $k_4$ - наклон в конце интервала с использованием $k_3$
    \end{itemize}
    Эти коэффициенты представляют собой взвешенные оценки производных на разных точках интервала.
    \item Обновляем значения переменных по формуле:
    \[
    y_{n+1} = y_n + \dfrac{1}{6}(k_1 + 2k_2 + 2k_3 + k_4)
    \]
    где коэффициенты 1:2:2:1 обеспечивают оптимальную точность метода 4-го порядка.
    \item Повторяем до достижения условия остановки ($y \leq 0$)
\end{enumerate}

\newpage
