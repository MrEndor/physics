\begin{center}
    \subsection{Метод 2: Моделирование с силой Гука}
\end{center}

В данном подходе шары рассматриваются как деформируемые объекты, подчиняющиеся закону Гука. 

При соприкосновении возникает упругая сила, пропорциональная деформации.

\begin{center}
    \textbf{Модель силы взаимодействия}
\end{center}

Пусть $\Delta$ -- величина взаимного перекрытия шаров:
\[
\Delta = 2R - |\vec{r}_2 - \vec{r}_1|
\]

При $\Delta > 0$ (шары перекрываются) возникает сила отталкивания:
\[
\vec{F} = -k\Delta \vec{n}
\]
где $k$ -- коэффициент жёсткости(сопротивления), $\vec{n}$ -- единичный вектор от центра первого шара к центру второго и наоборот.

\begin{center}
    \textbf{Система дифференциальных уравнений}
\end{center}

Для каждого шара применяем второй закон Ньютона:
\[
\begin{cases}
m_1 \ddfrac{d^2\vec{r}_1}{dt^2} = \vec{F}_{12} \\[1em]
m_2 \ddfrac{d^2\vec{r}_2}{dt^2} = \vec{F}_{21} = -\vec{F}_{12}
\end{cases}
\]

Покомпонентно:
\[
\begin{cases}
m_1 \ddfrac{d^2x_1}{dt^2} = F_x \\[1em]
m_1 \ddfrac{d^2y_1}{dt^2} = F_y \\[1em]
m_2 \ddfrac{d^2x_2}{dt^2} = -F_x \\[1em]
m_2 \ddfrac{d^2y_2}{dt^2} = -F_y
\end{cases}
\]

где сила взаимодействия:
\[
\begin{aligned}
F_x &= -k\Delta \frac{x_2 - x_1}{|\vec{r}_2 - \vec{r}_1|} \\
F_y &= -k\Delta \frac{y_2 - y_1}{|\vec{r}_2 - \vec{r}_1|}
\end{aligned}
\]

\begin{center}
    \textbf{Начальные условия}
\end{center}

В момент первого контакта ($t = 0$, $|\vec{r}_2 - \vec{r}_1| = 2R$):
\[
\begin{aligned}
\vec{r}_1(0) &= \vec{r}_{1,0}, & \quad \vec{v}_1(0) &= \vec{v}_{1,0} \\
\vec{r}_2(0) &= \vec{r}_{2,0}, & \quad \vec{v}_2(0) &= \vec{v}_{2,0}
\end{aligned}
\]

\begin{center}
    \textbf{Условие окончания взаимодействия}
\end{center}

Столкновение считается завершённым, когда:
\begin{enumerate}
    \item Шары расходятся: $\Delta \leq 0$
    \item Относительная скорость направлена от центра столкновения: $(\vec{v}_2 - \vec{v}_1) \cdot \vec{n} > 0$
\end{enumerate}

\begin{center}
    \textbf{Выбор параметров модели}
\end{center}

Коэффициент жёсткости $k$ выбирается достаточно большим для обеспечения:
\begin{itemize}
    \item Малого времени взаимодействия
    \item Малой максимальной деформации ($\Delta_{\max} \ll R$)
    \item Сохранения энергии с заданной точностью
\end{itemize}
