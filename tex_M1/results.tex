\begin{center}
    \section{Результаты и анализ}
    \subsection{Сравнение с теоретическими результатами}
\end{center}

Для идеального случая и модели с вязким трением численные результаты должны хорошо согласовываться с аналитическими решениями. Погрешность определяется шагом интегрирования $\Delta t$.
Для модели с лобовым сопротивлением аналитическое решение отсутствует, поэтому валидация проводится путём:
\begin{enumerate}
    \item \textbf{Проверки сходимости при уменьшении $\Delta t$} - убеждаемся, что решение стабилизируется при $\Delta t \to 0$
    \item \textbf{Сравнения с предельными случаями} ($\gamma \to 0, \ \beta \to 0$) - проверяем переход к идеальной модели
\end{enumerate}

\begin{center}
    \subsection{Влияние параметров на траекторию}
\end{center}

Исследуем влияние различных параметров:

\begin{enumerate}
    \item \textbf{Угол броска $\alpha$}: 
    \begin{itemize}
        \item В идеальном случае максимальная дальность при $\alpha = 45^\circ$ (симметричная парабола)
        \item При наличии сопротивления оптимальный угол меньше $45^\circ$ (обычно $35^\circ-40^\circ$)
        \item С увеличением коэффициента сопротивления оптимальный угол уменьшается
        \item При $\alpha = 90^\circ$ получаем вертикальный бросок с максимальной высотой
    \end{itemize}
    
    \item \textbf{Начальная скорость $v_0$}:
    \begin{itemize}
        \item Увеличение $v_0$ приводит к увеличению дальности и высоты
        \item Влияние сопротивления более значительно при больших $v_0$ (квадратичная зависимость)
    \end{itemize}
    
    \item \textbf{Коэффициент сопротивления}:
    \begin{itemize}
        \item Увеличение коэффициента уменьшает дальность полёта и максимальную высоту
        \item Траектория становится более крутой и асимметричной (восходящая ветвь положе)
    \end{itemize}
    
    \item \textbf{Масса тела $m$}:
    \begin{itemize}
        \item Увеличение массы уменьшает относительное влияние сопротивления
        \item Тяжёлые тела менее чувствительны к сопротивлению воздуха
        \item Лёгкие тела быстрее достигают предельной скорости
    \end{itemize}
\end{enumerate}

\textbf{Качественные особенности траекторий:}
\begin{itemize}
    \item \textit{Идеальный случай:} симметричная парабола
    \item \textit{Вязкое трение:} слегка асимметричная траектория, экспоненциальное затухание
    \item \textit{Лобовое сопротивление:} сильно асимметричная траектория, быстрое затухание скорости
\end{itemize}

\newpage
